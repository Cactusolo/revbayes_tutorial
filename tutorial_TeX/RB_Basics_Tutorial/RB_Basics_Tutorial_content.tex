\section{Basic \Rev Commands}

\subsection{Introduction}

This tutorial demonstrates the basic syntactical features of \RevBayes and \Rev and shows how to set up and perform an analysis on ``toy'' statistical models for linear regression. 
This tutorial focuses on explaining probabilistic graphical models and the language \Rev.
A good reference for probabilistic graphical models for Bayesian phylogenetic inference is given in \cite{Hoehna2014b}.
The statistical examples are borrowed from a fourth year statistics course taught in the fall term 2011 at Stockholm University.

The first section of this tutorial involves 
\begin{enumerate}
\item Creating different types of variables.
\item Learning about functions. 
\end{enumerate}

Then we will see how to perform statistical inference using \RevBayes and \Rev  by implementing a Monte Carlo algorithm. 
Finally, we will see how \RevBayes 's built-in functions vastly simplify this inference task.

All of the files for this analysis are provided for you, and you can run these without significant effort using the \cl{source()} function in the \RevBayes console:
{\tt \begin{snugshade*}
\begin{lstlisting}
source("RevBayes_scripts/basics.Rev")
\end{lstlisting}
\end{snugshade*}}
Nevertheless, you will learn more if you type in the commands directly.

Let's start with the basic concepts for the interactive use of \RevBayes with \Rev (the language of \RevBayes). 
You should try to execute the statements step by step, look at the output and try to understand what and why things are happening. 
We start with some simple concepts to get familiar and used to \RevBayes. 
By now you should have executed \RevBayes and you should see the command prompt waiting for input. 
The best exercise is to write these statements exactly in \RevBayes. 

\Rev is an interpreted language for statistical computing and analyses in evolutionary biology. Therefore, the basics are simple mathematical operations, such as 
{\tt \begin{snugshade*}
\begin{lstlisting}    
# Simple mathematical operators:
1 + 1                            # Addition
10 - 5                           # Subtraction
5 * 5                            # Multiplication
10 / 2                           # Division
2^3                              # Exponentiation
5%2                              # Modulo
\end{lstlisting}
\end{snugshade*}}
Just as a side note, you can also write multiple statements in the same line if you separate these by a semicolon (\cl{;}).
The statements will be executed as if you wrote each on a single line.
{\tt \begin{snugshade*}
\begin{lstlisting}    
1 + 1; 2 + 2                    # Multiple statements in one line
\end{lstlisting}
\end{snugshade*}}
    
Here you can see that comments always start with the hash symbol (\cl{\#}). 
Everything after the `\cl{\#}'-symbol will be ignored.
In addition to these simple mathematical operations, we provide some standard math functions which can be called by:
    
{\tt \begin{snugshade*}
\begin{lstlisting}    
# Math-Functions
exp(1)                           # exponential function
ln(1)                            # logarithmic function with natural base
sqrt(16)                         # square root function 
power(2,2)                       # power function: power(a,b) = a^b
\end{lstlisting}
\end{snugshade*}}
Notice that \Rev is case-sensitive. That means, \Rev distinguishes upper and lower case letter for both variable names and function names. For example, only the first of these two calls will work
{\tt \begin{snugshade*}
\begin{lstlisting}    
exp(1)                           # correct lower case name
Exp(1)                           # wrong upper case name
\end{lstlisting}
\end{snugshade*}}
Moreover, we provide functions for the common statistical distributions.
{\tt \begin{snugshade*}
\begin{lstlisting}    
# distribution functions
dexp(x=1,lambda=1)       # exponential distribution density function
qexp(0.5,1)              # exponential distribution quantile function
rexp(n=10,1)             # random draws from an exponential distribution
dnorm(-2.0,0.0,1.0)      # normal distribution density function
rnorm(n=10,0,1)          # random draws from a normal distribution
\end{lstlisting}
\end{snugshade*}}
You may have noticed that we sometimes provided labels of the arguments and sometimes not. 
You can always provide the argument labels and then \RevBayes will match the arguments based on the labels.
{\tt \begin{snugshade*}
\begin{lstlisting}    
dnorm(x=0.5,mean=0.0,sd=1)       # normal distribution density function
\end{lstlisting}
\end{snugshade*}}
If you do not provide the argument labels, then \RevBayes  will match the arguments by the best fitting types and the order in which you provided the arguments.
{\tt \begin{snugshade*}
\begin{lstlisting}    
dnorm(0.5,0.5,1)         # correct order
dnorm(0.5,1,0.5)         # mismatched order
\end{lstlisting}
\end{snugshade*}}
You may provide also just some arguments with labels and leave the other arguments without labels.
{\tt \begin{snugshade*}
\begin{lstlisting}    
dnorm(0.0,x=0.5,sd=1)    # partially labeled
\end{lstlisting}
\end{snugshade*}}
If you do not remember what the parameter name or parameter names of a function are, then you can simply type in the function name and \RevBayes will tell you the possible parameters with their names.
{\tt \begin{snugshade*}
\begin{lstlisting}    
dnorm
\end{lstlisting}
\end{snugshade*}}

\subsection{Variable Declaration}
The next, and very important feature of \RevBayes, is variable declaration. 
We have three types of (model) variables, namely constant, deterministic and stochastic variables, which represent the same three types of DAG nodes. 
Here we show how to construct the different variables and how they behave differently. 
First, we focus on the difference between constant and deterministic variables.

Let us begin by creating a constant variable with name \cl{a} and assigned the value 1 to it. 
The left arrow assignment (\cl{<-}) always creates a constant variable.
{\tt \begin{snugshade*}
\begin{lstlisting}    
# Variable assignment: constant and deterministic
a <- 1                           # assignment of constant node 'a'
\end{lstlisting}
\end{snugshade*}}
You see the value of 'a' by just typing in the variable name and pressing enter.
{\tt \begin{snugshade*}
\begin{lstlisting}    
a                                # printing the value of 'a'
\end{lstlisting}
\end{snugshade*}}
If you want to see which type of variable (constant, deterministic or stochastic) 'a' has, then call the structure function for it.
{\tt \begin{snugshade*}
\begin{lstlisting}    
str(a)                           # printing the structure information of 'a'
|*   _variable     = a
|*   _RevType      = Natural
|*   _RevTypeSpec  = [ Natural, Integer, RevObject ]
|*   _value        = 1
|*   _dagType      = Constant DAG node
|*   _children     = [  ]
|*   .methods = void function ()
\end{lstlisting}
\end{snugshade*}}
% Maybe not all of the above info is shown anymore?  I see only _variable, _dagType, and _children.
An additional quite useful built-in function in \RevBayes  is the \cl{type} function which gives you only the type information of the variable and thus is a subset of the \cl{str} function.
{\tt \begin{snugshade*}
\begin{lstlisting}    
type(a)                          # printing the type information of 'a'
|*    Natural
\end{lstlisting}
\end{snugshade*}}
Next, we create a deterministic variable \cl{b} using the \cl{:=} assignment computed by \cl{exp(a)} and another deterministic variable \cl{c} computed by \cl{ln(b)}. 
Deterministic variables are always created using the colon-equal assignment (\cl{:=}). 

{\tt \begin{snugshade*}
\begin{lstlisting}    
b := exp(a)                      # assignment of deterministic node 'b' with the exponential function with parameter 'a'
b                                # printing the value of 'b'
c := ln(b)                       # assignment of deterministic node 'c' with logarithmic function with parameter 'b'
c                                # printing the value of 'c'
\end{lstlisting}
\end{snugshade*}}
Again, you see the type of the variable and additional information such as which the parents and children are by calling the structure function on it.
{\tt \begin{snugshade*}
\begin{lstlisting}    
str(b)                           # printing the structure information of 'b'
\end{lstlisting}
\end{snugshade*}}
For example, see the difference to the creation of variable 'd', which is a constant variable.
{\tt \begin{snugshade*}
\begin{lstlisting}    
d <- ln(b)                       # assignment of constant node 'd' with the value if the logarithmic function with parameter 'b'
d                                # printing the value of 'd'
str(d)                           # printing the structure information of 'd'
\end{lstlisting}
\end{snugshade*}}
Currently, the variables \cl{c} and \cl{d} have the same value. 
We can check this using the equal comparison (\cl{==}).
{\tt \begin{snugshade*}
\begin{lstlisting}    
e := (c == d)			
e
\end{lstlisting}
\end{snugshade*}}
Now, if we assign a new value to variable \cl{a}, then naturally the value of \cl{a} changes. 
This has the consequence that all deterministic variables that use 'a' as a parameter, i.e., the variable \cl{b}, change their value automatically too.
{\tt \begin{snugshade*}
\begin{lstlisting}    
a <- 2                           # reassignment of variable a; every deterministic node which has 'a' as a parameter changes its value
a                                # printing the value of 'a'
b                                # printing the value of 'b'
c                                # printing the value of 'c'
d                                # printing the value of 'd'
e
\end{lstlisting}
\end{snugshade*}}
Since variable \cl{d} was a constant variable it did not change its value. 
This also means that \cl{e} is now false.

Finally, we show you how to create the third type of variables in \Rev: the stochastic variables. 
We will create a random variable \cl{x} from an exponential distribution with parameter \cl{lambda}.  
Stochastic assignments use the \cl{\rbdn} operation.
{\tt \begin{snugshade*}
\begin{lstlisting}     
# Variable assignment: stochastic
lambda <- 1                      # assign constant node 'lambda' with value '1'
x ~ dnExponential(lambda)        # create stochastic node with exponential distribution and parameter 'lambda'
\end{lstlisting}
\end{snugshade*}}
The value of \cl{x} is a random draw from the distribution. 
You can see the value and the probability (or log-probability) of the current value under the current parameter values by
{\tt \begin{snugshade*}
\begin{lstlisting}    
x                                # print value of stochastic node 'x'
x.probability()                  # print the probability if 'x'
x.lnProbability()                # print the log-probability if 'x'
str(x)                           # printing all the information of 'x'
\end{lstlisting}
\end{snugshade*}}
Similarly, we create a random variable \cl{y} from a normal distribution by
{\tt \begin{snugshade*}
\begin{lstlisting}    
mu <- 0
sigma <- 1
y ~ dnNorm(mu,sigma)	
y.probability()                  # print the probability of 'y'
y.lnProbability()                # print the log-probability if 'y'
str(y)                           # printing all the information of 'y'
\end{lstlisting}
\end{snugshade*}}

Variables that are not part of a model are assigned with \cl{=}, for example, \cl{i = 0}.

Now you know everything there is about creating the different types of variables and the different ways in which these variables behave.



\subsubsection{Simple variable manipulation and other types of assignments}
\Rev provides some convenience variable manipulation operations that are equivalent to variable manipulations in other programming languages such as C/C++, Java and Python.
You can increment (\cl{++}) and decrement (\cl{-\,-}) a variable.
The increment operation increases the current value of a variable by 1 and the decrement operation decreases the value by 1.
A post increment (\cl{a++}) increases the value after returning the value, that is, the old value is returned.
A pre increment (\cl{++a}) increases the value before returning the value, that is, the new value is returned.
{\tt \begin{snugshade*}
\begin{lstlisting}    
index <- 1
index++                          # post increment
++index                          # pre increment
index--                          # post decrement
--index                          # pre decrement
\end{lstlisting}
\end{snugshade*}}
Additionally, you can use addition (\cl{a += b}), subtraction (\cl{a -= b}), multiplication (\cl{a *= b}) and division (\cl{a /= b}) to an existing variable.
{\tt \begin{snugshade*}
\begin{lstlisting}    
index += 10                      # add 10 to the current value
index *= 2                       # double the current value
\end{lstlisting}
\end{snugshade*}}
These variable manipulations will come in very handy for indices of vectors/arrays.

\subsubsection{Vectors}
Common values in \RevBayes are of scalar types.
That means that not everything is a vector by default.
Instead, you can create a vector using three different ways.
First, you can call the vector function.
{\tt \begin{snugshade*}
\begin{lstlisting}    
v <- v(1,2,3)                    # create a vector
\end{lstlisting}
\end{snugshade*}}
Interestingly, we can use the same name for a variable as for a function: the variable \cl{v} and the function \cl{v(\ldots)}.
Both will still be fully functional and our interpreter checks if you asked for a function or a variable.

Second, you can use the square bracket notation.
{\tt \begin{snugshade*}
\begin{lstlisting}    
w <- [1,2,3]                     # create a vector
\end{lstlisting}
\end{snugshade*}}
And third, you can implicitly create the vector by assigning elements.
{\tt \begin{snugshade*}
\begin{lstlisting}    
z[1] <-1                         # implicit creation of a vector
z[2] <-2                   
z[3] <-3                  
\end{lstlisting}
\end{snugshade*}} 
The implicit creation does not need to instantiate the variable beforehand.
There are other useful built-in functions that produce vectors.
{\tt \begin{snugshade*}
\begin{lstlisting}    
1:10                             # range function
rep(10,1)                        # replicate an element n times
seq(1,20,2)                      # built a sequence from a to b by c
\end{lstlisting}
\end{snugshade*}} 

Vectors in \Rev belong to the class of objects that have methods.
You  can call a member method by
{\tt \begin{snugshade*}
\begin{lstlisting}    
x.<method name>(<arguments>)                 
\end{lstlisting}
\end{snugshade*}} 
You have seen two methods previously, \cl{probability} and \cl{lnProbability}.
If you don't remember what the methods were called, or if this object has any member methods, then you can get these by
{\tt \begin{snugshade*}
\begin{lstlisting}    
v.methods()                 
\end{lstlisting}
\end{snugshade*}} 
In general, this is very, very useful.
So for a vector we can get the size --- the number of elements --- by calling its member function:
{\tt \begin{snugshade*}
\begin{lstlisting}    
v.size()                 
\end{lstlisting}
\end{snugshade*}} 


\subsubsection{Control Structures}
In this next part we will learn about control structures in \Rev. 
The first control structure that we will look at is the \cl{for} loop.
A \cl{for} loop executes a single statement or a block of statements.
{\tt \begin{snugshade*}
\begin{lstlisting}    
# loops
for (<variable> in <set of value>) <single statement>
 
for (<variable> in <set of value>) 
   <single statement>

for (<variable> in <set of value>) {
   <multiple statements>
   <multiple statements>
   <multiple statements>
}
\end{lstlisting}
\end{snugshade*}}
The statement(s) will be executed for each value of variable of the \cl{for} loop.
A simple example is a \cl{for} loop that computes the sum of a sequence.
{\tt \begin{snugshade*}
\begin{lstlisting}    
sum <- 0
for (i in 1:100) {
   sum <- sum + i
}
sum
\end{lstlisting}
\end{snugshade*}}
Another example using a \cl{for} loop is the computation of the \href{http://en.wikipedia.org/wiki/Fibonacci_number}{Fibonacci number} for a given integer. 
{\tt \begin{snugshade*}
\begin{lstlisting}    
# Fibonacci series using a for loop
fib[1] <- 1
fib[2] <- 1
for (j in 3:10) {
   fib[j] <- fib[j - 1] + fib[j - 2]
}
fib
\end{lstlisting}
\end{snugshade*}}
We could also compute the Fibonacci numbers using a \cl{while} loop.
The \cl{while} loop continues to execute the statement(s) until the condition is wrong.
{\tt \begin{snugshade*}
\begin{lstlisting}    
# Fibonacci series using a while loop
fib[1] <- 1
fib[2] <- 1
j <- 3
while (j <= 10) {
   fib[j] <- fib[j - 1] + fib[j - 2]
   j++
}
fib
\end{lstlisting}
\end{snugshade*}}

\subsubsection{User Defined Functions}
In \Rev you can write your own functions as well.
The syntax for writing a function is:
{\tt \begin{snugshade*}
\begin{lstlisting}    
function <return value type> <function name> (<list of arguments>) { <statements> }
\end{lstlisting}
\end{snugshade*}}
As a simple example, let's write a function that computes the square of a number.
We expect that the function takes in any real number.
The type of real number is \cl{Real}.
Since the square is always a positive real number, we choose the return to be \cl{RealPos}
{\tt \begin{snugshade*}
\begin{lstlisting}    
# simple square function
function RealPos square ( Real x ) { x * x }
\end{lstlisting}
\end{snugshade*}}
Now we can call our own function the same way as we call other already built-in functions in \RevBayes.
{\tt \begin{snugshade*}
\begin{lstlisting}    
a <- square(5.0)
a
\end{lstlisting}
\end{snugshade*}}
As an exercise, let's write a function that computes the factorial of a natural number.
{\tt \begin{snugshade*}
\begin{lstlisting}    
# function for computing the factorial
function Natural fac(i) {
   if (i > 1) {
      return i * fac(i-1)
   } else {
      return 1
   }
}
b <- fac(6)
b
\end{lstlisting}
\end{snugshade*}}
Here you see that within your own function you can call your function as well, which is commonly called recursive function calls.

Now let us write a recursive function for the sum of numbers which we computed before using a \cl{for} loop.
{\tt \begin{snugshade*}
\begin{lstlisting}    
# function for computing the sum
function Integer sum(Integer j) {
   if (j > 1) {
      return j + sum(j-1)
   } else {
      return 1
   }
}
c <- sum(100)
c
\end{lstlisting}
\end{snugshade*}}
We can do the same for our favorite example, the Fibonacci series.
{\tt \begin{snugshade*}
\begin{lstlisting}    
# function for computing the fibonacci series
function Integer fib(Integer k) {
   if (k > 1) {
      return fib(k-1) + fib(k-2)
   } else {
      return k
   }
}
d <- fib(6)
d
\end{lstlisting}
\end{snugshade*}}
Now that should be enough to get you going with our first example analyses.




\newpage
\FloatBarrier
\section{Exercise: Poisson Regression Model for Airline Fatalities}

This exercise will demonstrate how to approximate the posterior distribution of some parameters using a simple Metropolis algorithm. 
The focus here lies in the Metropolis algorithm, Bayesian inference, and model specification---but not in the model or the data. 
After completing this computer exercise, you should be familiar with the basic Metropolis algorithm, analyzing output generated from a MCMC algorithm, and performing standard Bayesian inference.

\subsection{Model and Data}
We will use the data example from \cite{Gelman2003}.
A summary is given in Table~\ref{tab:airlineFatalities}.
\begin{table}[!hbtp]
\caption{Airline fatalities from 1976 to 1985. Reproduced from \cite[][Table~2.2 on p.~69]{Gelman2003}.}
\label{tab:airlineFatalities}
\smallskip
\centering
\begin{tabular}{ l | r r r r r r r r r r }
  \hline                       
  Year & 1976 & 1977 & 1978 & 1979 & 1980 & 1981 & 1982 & 1983 & 1984 & 1985 \\
  Fatalities & 24 & 25 & 31 & 31 & 22 & 21 & 26 & 20 & 16 & 22\\
  \hline  
\end{tabular}
\end{table}

These data can be loaded into \RevBayes by typing:
{\tt \begin{snugshade*}
\begin{lstlisting}    
observed_fatalities <- v(24,25,31,31,22,21,26,20,16,22)
\end{lstlisting}
\end{snugshade*}}

The model is a \href{http://en.wikipedia.org/wiki/Poisson_regression}{Poisson regression} model with parameters $\alpha$ and $\beta$
\begin{equation*}
y \sim \text{Poisson}(\exp(\alpha+\beta*x))
\end{equation*} 
where $y$ is the number of fatal accidents in year $x$. 
For simplicity, we choose uniform priors for $\alpha$ and $\beta$.
\begin{eqnarray*}
\alpha & \sim & \text{Uniform}(-10,10)\\
\beta &  \sim & \text{Uniform}(-10,10)
\end{eqnarray*}
The probability density can be computed in \RevBayes for a single year by
{\tt \begin{snugshade*}
\begin{lstlisting}    
dpoisson(y[i],exp(alpha+beta*x[i]))
\end{lstlisting}
\end{snugshade*}}

\subsection{Problems}

\subsubsection{Metropolis Algorithm}%

The source file for this sub-exercise \cl{airline\_fatalities\_part1.Rev}.

Let us construct a Metropolis algorithm that simulates from the posterior distribution $P(\alpha,\beta|y)$. 
We will construct this algorithm explicitly, without using the high-level functions existing in \RevBayes  to perform MCMC. 
In the next section, we will repeat the same analysis, this time using the high-level functions.
(More background on MCMC is provided in the \href{https://github.com/revbayes/revbayes_tutorial/raw/master/tutorial_TeX/RB_MCMC_Intro_Tutorial/RB_MCMC_Intro_Tutorial.pdf}{Introduction to Markov Chain Monte Carlo Algorithms tutorial}.)
 
For simplicity of the calculations you can ``normalize'' the years, e.g. 
{\tt \begin{snugshade*}
\begin{lstlisting}    
x <- 1976:1985 - mean(1976:1985)
\end{lstlisting}
\end{snugshade*}}

A common proposal distribution for $\alpha^{\prime} \sim P(\alpha[i-1])$ is the normal distribution with mean $\mu = \alpha[i-1]$ and standard deviation $\sigma = \delta_\alpha$:
% \begin{equation}
% \alpha^{\prime} \sim \text{norm}(alpha[i-1],delta\_alpha)
% \end{equation}

{\tt \begin{snugshade*}
\begin{lstlisting}    
alpha_prime <- rnorm(1,alpha[i-1],delta_alpha)
\end{lstlisting}
\end{snugshade*}}
A similar distribution should be used for $\beta^{\prime}$. 
{\tt \begin{snugshade*}
\begin{lstlisting}    
delta_alpha <- 1.0
delta_beta <- 1.0
\end{lstlisting}
\end{snugshade*}}
After you look at the output of the MCMC (later), play around to find appropriate values for $\delta_{\alpha}$ and $\delta_{\beta}$.

Now we need to set starting values for the MCMC algorithm.
Usually, these are drawn from the prior distribution, but sometimes if the prior is very uninformative, then these parameter values result in a likelihood of 0.0 (or log-likelihood of -Inf).
{\tt \begin{snugshade*}
\begin{lstlisting}    
alpha[1] <- -0.01     # you can also use runif(-1.0,1.0)
beta[1] <- -0.01      # you can also use runif(-1.0,1.0)
\end{lstlisting}
\end{snugshade*}}
Next, create some output for our MCMC algorithm.
The output will be written into a file that can be read into \R or Tracer \citep{Rambaut2011}.
{\tt \begin{snugshade*}
\begin{lstlisting}    
# create a file output
write("iteration","alpha","beta",file="airline_fatalities.log")
write(0,alpha[1],beta[1],file="airline_fatalities.log",append=TRUE)
\end{lstlisting}
\end{snugshade*}}
Note that we need a first iteration with value 0 so that Tracer can load in this file.

Finally, we set up a \cl{for} loop over each iteration of the MCMC.
{\tt \begin{snugshade*}
\begin{lstlisting}    
for (i in 2:10000) {
\end{lstlisting}
\end{snugshade*}}
Within the \cl{for} loop we propose new parameter values.
{\tt \begin{snugshade*}
\begin{lstlisting}    
    alpha_prime <- rnorm(1,alpha[i-1],delta_alpha)[1]
    beta_prime <- rnorm(1,beta[i-1],delta_beta)[1]
\end{lstlisting}
\end{snugshade*}}
For the newly proposed parameter values we compute the prior ratio.
In this case we know that the prior ratio is 0.0 as long as the new parameters are within the limits.
{\tt \begin{snugshade*}
\begin{lstlisting}    
    ln_prior_ratio <- dunif(alpha_prime,-10.0,10.0,log=TRUE) + dunif(beta_prime,-10.0,10.0,log=TRUE) - dunif(alpha[i-1],-10.0,10.0,log=TRUE) - dunif(beta[i-1],-10.0,10.0,log=TRUE)
\end{lstlisting}
\end{snugshade*}}
Similarly, we compute the likelihood ratio for each observation.
{\tt \begin{snugshade*}
\begin{lstlisting}    
    ln_likelihood_ratio <- 0
    for (j in 1:x.size() ) {
       lambda_prime <- exp( alpha_prime + beta_prime * x[j] )
       lambda <- exp( alpha[i-1] + beta[i-1] * x[j] )
       ln_likelihood_ratio += dpoisson(observed_fatalities[j],lambda_prime) - dpoisson(observed_fatalities[j],lambda)
    }
    ratio <- ln_prior_ratio + ln_likelihood_ratio
\end{lstlisting}
\end{snugshade*}}
And finally we accept or reject the newly proposed parameter values with probability \cl{ratio}.
{\tt \begin{snugshade*}
\begin{lstlisting}    
    if ( ln(runif(1)[1]) < ratio) {
       alpha[i] <- alpha_prime
       beta[i] <- beta_prime
    } else {
       alpha[i] <- alpha[i-1]
       beta[i] <- beta[i-1]
    }
\end{lstlisting}
\end{snugshade*}}
Then we log the current parameter values to the file by appending the file.
{\tt \begin{snugshade*}
\begin{lstlisting}    
    # output to a log-file
    write(i-1,alpha[i],beta[i],file="airline_fatalities.log",append=TRUE)
 }
\end{lstlisting}
\end{snugshade*}}
As a quick summary you can compute the posterior mean of the parameters.
{\tt \begin{snugshade*}
\begin{lstlisting}    
mean(alpha)
mean(beta)
\end{lstlisting}
\end{snugshade*}}
You can also load the file into \R or Tracer to analyze the output.


In this section of the first exercise we wrote our own little Metropolis algorithm in \Rev.
This becomes very cumbersome, difficult and slow if we'ld need to do this for every model.
Here we wanted to show you only the basic principle of any MCMC algorithm.
In the next section we will use the built-in MCMC algorithm of \RevBayes.




\subsubsection{MCMC analysis using the built-in algorithm in \RevBayes}
Before starting with this new approach it would be good if you either start a new \RevBayes session or clear all previous variables using the \cl{clear} function.
Currently we may have some minor memory problems and if you get stuck it may help to restart \RevBayes.

We start by loading in the data to \RevBayes.
{\tt \begin{snugshade*}
\begin{lstlisting} 
observed_fatalities <- v(24,25,31,31,22,21,26,20,16,22)
x <- 1976:1985 - mean(1976:1985)
\end{lstlisting}
\end{snugshade*}}
Then we create the parameters with their prior distributions.
{\tt \begin{snugshade*}
\begin{lstlisting} 
alpha ~ dnUnif(-10,10) 
beta ~ dnUnif(-10,10)
\end{lstlisting}
\end{snugshade*}}
It may be good to set some reasonable starting values especially if you choose a very uninformative prior distribution.
If by chance you had starting values that gave a likelihood of -Inf, then \RevBayes will try several times to propose new starting values drawn from the prior distribution.
{\tt \begin{snugshade*}
\begin{lstlisting} 
# let us use reasonable starting value
alpha.setValue(0.0)
beta.setValue(0.0)
\end{lstlisting}
\end{snugshade*}}
Our next step is to set up the moves.
Moves are algorithms that propose new values and know how to reset the values if the proposals are rejected.
We use the same sliding window move as we implemented above by ourselves.
{\tt \begin{snugshade*}
\begin{lstlisting} 
mi <- 0
moves[mi++] = mvSlide(alpha)
moves[mi++] = mvSlide(beta)
\end{lstlisting}
\end{snugshade*}}
Then we set up the model.
This means we create a stochastic variable for each observation and clamp its value with the observed data.
{\tt \begin{snugshade*}
\begin{lstlisting} 
for (i in 1:x.size() ) {
    lambda[i] := exp( alpha + beta * x[i] )
    y[i] ~ dnPoisson(lambda[i])
    y[i].clamp(observed_fatalities[i])
}
\end{lstlisting}
\end{snugshade*}}
We can now create the model by pulling up the model graph from any variable that is connected to our model graph.
{\tt \begin{snugshade*}
\begin{lstlisting} 
mymodel = model( alpha )
\end{lstlisting}
\end{snugshade*}}
We also need some monitors that report the current values during the MCMC run.
We create two monitors, one printing all numeric non-constant variables to a file and one printing some information to the screen.
{\tt \begin{snugshade*}
\begin{lstlisting} 
monitors[1] = mnModel(filename="output/airline_fatalities.log",printgen=10, separator = "	")
monitors[2] = mnScreen(printgen=10, alpha, beta)
\end{lstlisting}
\end{snugshade*}}
Finally we create an MCMC object.
The MCMC object takes in a model object, the vector of monitors and the vector of moves.
{\tt \begin{snugshade*}
\begin{lstlisting} 
mymcmc = mcmc(mymodel, monitors, moves)
\end{lstlisting}
\end{snugshade*}}
On the MCMC object we call its member method \cl{run} to run the MCMC.
{\tt \begin{snugshade*}
\begin{lstlisting} 
mymcmc.run(generations=3000)
\end{lstlisting}
\end{snugshade*}}
And now we are done {\LARGE \smiley}


\subsubsection{Posterior Distribution of $\alpha$ and $\beta$}
 
Report the posterior mean and 95\% credible intervals for $\alpha$ and $\beta$. 
Additionally, plot the posterior distribution of $\alpha$ and $\beta$ by plotting a histogram of the samples. 
% You can use the \R function
% \RCode{
% hist(alpha,nclass=20)
% }
%For more information consult the \R help about the histogram function. To export the figure you need to use commands similar to
%\RCode{
%png("myFigure.png")
%}
%then your commands for printing the figure (e.g. hist(alpha,nclass=20)) and then
%\RCode{
%dev.off()
%}

Plot the curve of $m(x) = \text{E}[\exp(\alpha+\beta*x)|y]$ for $x = [1976,1985]$. 
You can generate draws from the posterior distribution of the expected value for a specific $x$ by recording the current expected value at a iteration $i$ of the Metropolis algorithm $m\_sample(x)[i] = \text{E}[\exp(\alpha[i]+\beta[i]*x)|y]$ and taking the mean of those samples (\cl{m(x) = \text{mean}(m\_sample(x))}) afterwards. Since \RevBayes provides you with the samples of $m(x) = \text{E}[\exp(\alpha+\beta*x)|y] = \lambda_x$ you can simply plot these posterior curves.
 
%A plot of the posterior mean curve $m(x)=E(\exp(\alpha+\beta*x)|y)$ over a suitable range. 
%A few draws from the posterior curve, i.e. \cl{exp(alpha[i]+beta[i]*x)} for a few i:s would also be nice.
%(These are somewhat cumbersome to do in R, you may need to present sample code).

 
Produce a histogram of the predictive distribution of the number of fatalities in 2014 and estimate the posterior mean. 
The predictive distribution can be approximated simultaneously with the Metropolis algorithm. 
This means, for any iteration $i$ you simulate draws from the conditional distribution for $x = 2014$ and the current values of $\alpha[i]$ and $\beta[i]$.
 
Estimate the distribution of the mean of the posterior predictive distribution of the the number of fatalities in 2014. 
Therefore, let us denote the expected value of the posterior distribution by $\mu$. 
Since we do not know this value $\mu$ exactly, we can follow the Bayesian approach and associate a probability for each value $m$ as being the true expected value of the posterior distribution, given the observations $y$ ($P(m = \mu|y)$).
You can approximate this distribution by recording the expected value for the number of fatalities in 2014 ($\text{E}[\exp(\alpha+\beta*x)|y]$) in each iteration $i$ of the Metropolis algorithm. 
Plot a histogram of the expected values, compute the mean of the expected values and compare it to the previously obtained estimate of the mean of the posterior predictive distribution.
 
Follow the same approach as for the posterior predictive distribution for $x = 2014$, but this time for $x = 2016$ and estimate the probability of no fatality.
 
 
 
 

\newpage
\FloatBarrier
\section{Exercise: Poisson Regression Model for Coal-mine Accidents}
 
We will analyze a dataset coal-mine accidents.
The values are the dates of major (more than 10 casualties) coal-mining disasters in the UK from 1851 to 1962. 


\subsection{A model for disasters}

A common model for the number of events that occur over a period of time is a Poisson process, in which the numbers of events in disjoint time-intervals are independent and Poisson-distributed. 
We will discretize and look at the yearly number of accidents. 

In order to take into account the possible change of rate, we will allow for different rates before and after year $\theta$, where $\theta$ is unknown to us. 
Thus, the observation distribution of our model is 
$y_t \sim Poisson(\lambda_t)$ with $t = 1851,\ldots,1962$ and
\begin{eqnarray*}
\lambda_t & = & \begin{cases}
\beta & \mbox{if } t < \theta \\
\gamma & \mbox{if } t \geq \theta
\end{cases}
\end{eqnarray*}
Thus, the rate $\lambda_t$ is defined by three unknown parameters: $\beta$, $\gamma$ and $\theta$. A hierarchical choice of priors is given by
\begin{eqnarray*}
 \eta & \sim & Gamma(10.0;20.0) \\ 
 \beta & \sim & Gamma(2.0;\eta) \\
 \gamma & \sim &Gamma(2.0;\eta) \\
 \theta & \sim & Uniform(1852,\ldots,1962)
\end{eqnarray*}
which brings an additional parameter $\eta$ in the model. 
For $\theta$ we have used a uniform prior over the years, but excluded year 1851 in order to make sure at least one year has rate $\beta$.
The hierarchical prior carries the belief that $\beta$ and $\gamma$ are somewhat similar in size,
since they both depend on $\eta$. 

\subsection*{The model in \Rev}

We start as usual by loading in the data.
{\tt \begin{snugshade*}
\begin{lstlisting} 
observed_fatalities <-  v(4, 5, 4, 1, 0, 4, 3, 4, 0, 6, 3, 3, 4, 0, 2, 6, 3, 3, 5, 4, 5, 3, 1, 4, 4, 1, 5, 5, 3, 4, 2, 5, 2, 2, 3, 4, 2, 1, 3, 2, 2, 1, 1, 1, 1, 3, 0, 0, 1, 0, 1, 1, 0, 0, 3, 1, 0, 3, 2, 2, 0, 1, 1, 1, 0, 1, 0, 1, 0, 0, 0, 2, 1, 0, 0, 0, 1, 1, 0, 2, 3, 3, 1, 1, 2, 1, 1, 1, 1, 2, 3, 3, 0, 0, 0, 1, 4, 0, 0, 0, 1, 0, 0, 0, 0, 0, 1, 0, 0, 1, 0, 1)
year <- 1851:1962
\end{lstlisting}
\end{snugshade*}}
In \Rev we specify this prior choice by
{\tt \begin{snugshade*}
\begin{lstlisting} 
eta ~ dnGamma(10.0,20.0)
beta ~ dnGamma(2.0,eta)
gamma ~ dnGamma(2.0,eta)
theta ~ dnUnif(1852.0,1962.0)
\end{lstlisting}
\end{snugshade*}}
Then we select moves for each parameter.
For the rate parameters --- which are defined only on the positive real line --- we choose a scaling move.
Only for \cl{theta} we choose the sliding window proposal.
{\tt \begin{snugshade*}
\begin{lstlisting} 
mi <- 0
moves[mi++] = mvScale(eta)
moves[mi++] = mvScale(beta)
moves[mi++] = mvScale(gamma)
moves[mi++] = mvSlide(theta)
\end{lstlisting}
\end{snugshade*}}
Then, we set up the model by computing the conditional rate of the Poisson distribution, creating random variables for each observation and attaching (clamping) data to the variables.
{\tt \begin{snugshade*}
\begin{lstlisting} 
for (i in 1:year.size() ) {
    rate[i] := ifelse(theta > year[i], beta, gamma)
    y[i] ~ dnPoisson(rate[i])
    y[i].clamp(observed_fatalities[i])
}
\end{lstlisting}
\end{snugshade*}}
Finally, we create the model object from the variables, add some monitors and run the MCMC algorithm.
{\tt \begin{snugshade*}
\begin{lstlisting} 
mymodel = model( theta )

monitors[1] = mnModel(filename="output/coal_accidents.log",printgen=10, separator = "	")
monitors[2] = mnScreen(printgen=10, eta, lambda, gamma, theta)

mymcmc = mcmc(mymodel, monitors, moves)

mymcmc.run(generations=3000)
\end{lstlisting}
\end{snugshade*}}




%\subsection*{Output analysis}
%
%Run the algorithm for say N = 10000 iterations or more. 
%
%a) In 1872, legislation on safety in mines was strengthened. 
%In 1878 and 1897 legislation on liability of employers for accidents was strengthened. Approximate the probability that the change occurred in the year after either of these changes (expect small numbers).
%b) How could the information given in a) be used to construct a prior for ?


%\subsection*{Posterior curves}

%We will now look at ways of visualising the posterior of the rate-function t, t =
%1851; : : : ; 1962. First we plot data together with a few posterior draws:
%par(mfrow=c(1,1))
%plot(year,y)
%I <- (1:10)*500
%for (i in I){
%points(c(1851,theta[i],1962),c(lambda[i],gamma[i],gamma[i]),type="s")
%}
%This gives a visual impression of the posterior uncertainty involved. As a point-estimate,
%we start with the mean curve, i.e. t ! m(t) = E(tjy). This can be approximated as
%follows
%t<-1851:1962
%for (i in 1:112){
%m[i]<-mean(lambda*(theta>=t[i])+gamma*(theta<t[i]))
%}
%and added to the plot in green
%lines(t,m,lwd=4,col="green")
%An alternative point-estimate of t is the mode, i.e. the curve that maximises the posterior probability. In order to find this you need the stored values of the log-posterior
%in lp: we want the index i that gives the largest lp[i]. This is provided in R by
%i <- which.max(lp)
%next, plot in red by
%points(c(1851,theta[i],1962),c(lambda[i],gamma[i],gamma[i]),type="s",lwd=4,col="red")


\bigskip
\subsection{Batch Mode}

If you wish to run this exercise in batch mode, the files are provided for you. 

You can carry out these batch commands by providing the file name when you execute the \cl{rb} binary in your unix terminal (this will overwrite all of your existing run files).
\exs{\cl{\$ rb RevBayes\_scripts airline\_fatalities\_part1.Rev}}
\exs{\cl{\$ rb RevBayes\_scripts airline\_fatalities\_part2.Rev}}
\exs{\cl{\$ rb RevBayes\_scripts coalmine\_accidents.Rev}}


\bibliographystyle{sysbio}
\bibliography{\GlobalResourcePath refs}
