
\section{Introduction}

\citet{Ronquist2012a}

\subsection{Models}

\subsubsection{Sequence Evolution}

Point to other tutorials (e.g., GTR stuff)

\subsubsection{Morphological Character Change}

Mk models and ascertainment bias

\subsubsection{Lineage-Specific Substitution Rates}

Clocks \citep{Zuckerkandl1962} and relaxing them 

\subsubsection{Lineage Diversification and Sampling}

Birth-death processes and FBD

\section{Prerequisites}

What do you need to know before doing this?

\subsection{Requirements}
We assume that you have read and hopefully completed the following tutorials:
\begin{itemize}
\item RB\_Getting\_Started
\item RB\_Basics\_Tutorial
\end{itemize}
Note that the RB\_Basics\_Tutorial introduces the basic syntax of \Rev but does not cover any phylogenetic models.
You may skip the RB\_Basics\_Tutorial if you have some familiarity with \R.
We tried to keep this tutorial very basic and introduce all the language concepts on the way.
You may only need the RB\_Basics\_Tutorial for a more in-depth discussion of concepts in \Rev.


%%%%%%%%
%%   Data   %%
%%%%%%%%
\section{Data and files}

We provide the data file(s) which we will use in this tutorial.
You may want to use your own data instead.
In the \cl{data} folder, you will find the following files
\begin{itemize}
\item \cl{bears\_extant\_cytb.nex}: description of file/data (we also need more descriptive file names).
\end{itemize}



\section{Exercise: Title}

\bigskip
\subsection{Getting Started}

We will complete this analysis in \RevBayes by entering the \Rev code interactively. 


\bigskip

\subsection{Loading the Data}

\noindent \\ \impmark Download data and output files (if you don't have them already) from: \\ \href{http://revbayes.github.io/tutorials.html}{http://revbayes.github.io/tutorials.html}


First load in the sequences using the \cl{readDiscreteCharacterData()} function. 
{\tt \begin{snugshade*}
\begin{lstlisting}
data <- readDiscreteCharacterData("data/bears_cytb.nex") 
\end{lstlisting}
\end{snugshade*}}
Executing these lines initializes the data matrix as the respective \Rev variables. 
To report the current value of any variable, simply type the variable name and press enter. For the \cl{data} matrix, this provides information about the alignment:
{\tt \begin{snugshade*}
\begin{lstlisting}
data
|*   DNA character matrix with 23 taxa and 1141 characters
|*   =====================================================
|*   Origination:                      bears_cytb.nex
|*   Number of taxa:                   10
|*   Number of included taxa:          10
|*   Number of characters:             1000
|*   Number of included characters: 1000
|*   Datatype:                         DNA
\end{lstlisting}
\end{snugshade*}}
%% This is what output looks like

%\begin{figure}[htbp!]
%\centering
%\fbox{\includegraphics[width=0.8\textwidth]{\ResourcePath figures/figure.pdf}}
%\caption{\small example figure.}
%\label{fig:example}
%\end{figure}

\bibliographystyle{sysbio}
\bibliography{\GlobalResourcePath refs}
