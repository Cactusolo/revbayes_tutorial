\section{Partitioning by Gene Region}\label{secByGene}

The uniform model used in the previous section assumes that all sites in the alignment evolved under the same process described by a shared tree, branch length proportions, and parameters of the GTR+$\Gamma$ substitution model.
However, our alignment contains two distinct gene regions---cytB and cox2---so we may wish to explore the possibility that the substitution process differs between these two gene regions.
This requires that we first specify the data partitions corresponding to these two genes, then define an independent substitution model for each data partition. 

First, we'll clear the workspace of all declared variables
{\tt \begin{snugshade*}
\begin{lstlisting}
clear()
\end{lstlisting}
\end{snugshade*}}

Since we wish to avoid individually specifying each parameterster of the GTR+$\Gamma$ model for each of our data partitions, we can \textit{loop} over our datasets and create vectors of nodes.
To do this, we begin by creating a vector of data file names:
{\tt \begin{snugshade*}
\begin{lstlisting}
filenames <- v("data/primates_cox2.nex", "data/primates_cytb.nex")
\end{lstlisting}
\end{snugshade*}}

Set a variable for the number of partitions:
{\tt \begin{snugshade*}
\begin{lstlisting}
n_parts <- filenames.size()
\end{lstlisting}
\end{snugshade*}}

And create a vector of data matrices called \cl{data}:
{\tt \begin{snugshade*}
\begin{lstlisting}
for (i in 1:n_parts){
   data[i] = readDiscreteCharacterData(filenames[i])
}
\end{lstlisting}
\end{snugshade*}}

Next, we can initialize some important variables. This does require, however, that both of our alignments have the same number of species and matching tip names.
{\tt \begin{snugshade*}
\begin{lstlisting}
n_species <- data[1].ntaxa()
names <- data[1].names()
n_branches <- 2 * n_species - 3
\end{lstlisting}
\end{snugshade*}}

{\tt \begin{snugshade*}
\begin{lstlisting}
mvi <- 1
mni <- 1
\end{lstlisting}
\end{snugshade*}}


\subsection{Specify the Parameters by Looping Over Partitions}

We can avoid creating unique names for every node in our model if we use a \cl{for} loop to iterate over our partitions. Thus, we will only have to type in our entire GTR+$\Gamma$ model parameters once. 
This will produce a vector for each of the unlinked parameters --- e.g., there will be a vector of \cl{alpha} nodes where the stochastic node for the first partition (cytB) will be \cl{alpha[1]} and the stochastic node for the second partition (cox2) will be called \cl{alpha[2]}.

The script for the model, {\tt RevBayes\_scripts/model\_PS1.Rev}, creates the model parameters for each partition in one large loop.
Here, we will split the loop into smaller parts to achieve the same end.

First, we will create the GTR rate matrix for partition $i$ by first creating exchangeability rates
{\tt \small \begin{snugshade*}
\begin{lstlisting}
for (i in 1:n_parts) {
  er_prior[i] <- v(1,1,1,1,1,1)
  er[i] ~ dnDirichlet(er_prior[i])
  moves[mvi++] = mvSimplexElementScale(er[i], alpha=10, tune=true, weight=3) 
}
\end{lstlisting}
\end{snugshade*}}

and stationary frequencies

{\tt \small \begin{snugshade*}
\begin{lstlisting}
for (i in 1:n_parts) {
  pi_prior[i] <- v(1,1,1,1)
  pi[i] ~ dnDirichlet(pi_prior[i])
  moves[mvi++] = mvSimplexElementScale(pi[i], alpha=10, tune=true, weight=2)
}
\end{lstlisting}
\end{snugshade*}}

then passing those parameters into a rate matrix function

{\tt \small \begin{snugshade*}
\begin{lstlisting}
for (i in 1:n_parts) {
  Q[i] := fnGTR(er[i],pi[i]) 
}
\end{lstlisting}
\end{snugshade*}}

which states the rate matrix ({\tt Q[i]}) for partition $i$ is determined by the exchangeability rates ({\tt er[i]}) and stationary frequencies ({\tt pi[i]}) also defined for partition $i$.
Following this format, we construct the remainin partition parameters: the $+\Gamma$ mixture model

{\tt \small \begin{snugshade*}
\begin{lstlisting}
for (i in 1:n_parts) {
  alpha_prior_min[i] <- 0.1
  alpha_prior_max[i] <- 50.0
  alpha[i] ~ dnUnif( alpha_prior_min[i], alpha_prior_max[i] )
  norm_gamma_rates[i] := fnDiscretizeGamma( alpha[i], alpha[i], 4, false )
  moves[mvi++] = mvScale(alpha[i], lambda=0.8, tune=true, weight=3.0)
}
\end{lstlisting}
\end{snugshade*}}

the $+I$ invariant sites model

{\tt \small \begin{snugshade*}
\begin{lstlisting}
for (i in 1:n_parts) {
  pinvar[i] ~ dnBeta(1,1)
  moves[mvi++] = mvScale(pinvar[i], lambda=0.1, tune=true, weight=2.0)
  moves[mvi++] = mvSlide(pinvar[i], delta=0.1, tune=true, weight=2.0)
}
\end{lstlisting}
\end{snugshade*}}

and the per-partition molecular clock

{\tt \small \begin{snugshade*}
\begin{lstlisting}
for (i in 1:n_parts) {
  part_rate_mult[i] ~ dnExponential( 1.0 )
  moves[mvi++] = mvScale(part_rate_mult[i], lambda=1.0, tune=true, weight=2.0)
}
\end{lstlisting}
\end{snugshade*}}

Our two genes evolve under different GTR rate matrices with different mean-one gamma distributions on the site rates.
However, we do assume that they share a single topology and set of branch lengths.
{\tt \begin{snugshade*}
\begin{lstlisting}
speciation ~ dnExponential(10.)
extinction ~ dnExponential(10.)

# rescaling mv on speciation and extinction rates
moves[mvi++] = mvScale(speciation, lambda=1, tune=true, weight=3.0)
moves[mvi++] = mvScale(extinction, lambda=1, tune=true, weight=3.0)

sampling_fraction <- 23. / 270 # 23 out of the ~ 270 primate species

psi ~ dnBDP(lambda=speciation, mu=extinction, rho=sampling_fraction, rootAge=75, names=names)

# moves on the tree
moves[mvi++] = mvNNI(psi, weight=10.0)
moves[mvi++] = mvFNPR(psi, weight=5.0)
moves[mvi++] = mvNodeTimeSlideUniform(psi, weight=10.0)
\end{lstlisting}
\end{snugshade*}}

which is the same as was specified for {\tt model\_PS0.Rev}.

Since we have a rate matrix and a site-rate model for each partition, we must create a phylogenetic CTMC for each gene. 
Additionally, we must fix the values of these nodes by attaching their respective data matrices.
These two nodes are linked by the \cl{psi} node and their log-likelihoods are added to get the likelihood of the whole DAG.
{\tt \begin{snugshade*}
\begin{lstlisting}
for (i in 1:n_parts) {
  phyloSeq[i] ~ dnPhyloCTMC(tree=psi, Q=Q[i], branchRates=part_rate_mult[i], siteRates=norm_gamma_rates[i], pInv=pinvar[i], type="DNA")
  phyloSeq[i].clamp(data[i])
}
\end{lstlisting}
\end{snugshade*}}

The remaining steps should be familiar:
wrap the model components in a model object

{\tt \begin{snugshade*}
\begin{lstlisting}
mymodel = model(psi)
\end{lstlisting}
\end{snugshade*}}

create the monitors

{\tt \begin{snugshade*}
\begin{lstlisting}
monitors[mni++] = mnScreen(er, pi, alpha, norm_gamma_rates, pinvar, speciation, extinction, part_rate_mult, printgen=100)
monitors[mni++] = mnFile(psi, filename="output/PS1.tre", printgen=100)
monitors[mni++] = mnFile(er[1], pi[1], alpha[1], norm_gamma_rates[1], pinvar[1], part_rate_mult[1], er[2], pi[2], alpha[2], norm_gamma_rates[2], pinvar[2], part_rate_mult[2], speciation, extinction, filename="output/PS1.params.txt",printgen=10)
\end{lstlisting}
\end{snugshade*}}

configure and run the MCMC analysis

{\tt \begin{snugshade*}
\begin{lstlisting}
mymcmc = mcmc(mymodel, moves, monitors)
mymcmc.burnin(10000,100)
mymcmc.run(30000)
\end{lstlisting}
\end{snugshade*}}

and summarize the posterior density of trees with a MAP tree

{\tt \begin{snugshade*}
\begin{lstlisting}
treetrace = readTreeTrace("output/PS1.tre")
treetrace.summarize()
mapTree(treetrace,"output/PS1_map.tre")
\end{lstlisting}
\end{snugshade*}}
