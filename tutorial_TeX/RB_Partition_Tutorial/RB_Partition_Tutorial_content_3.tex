\section{Partitioning by Codon Position and by Gene}\label{secExtremeP}

Because of the genetic code, we often find that different positions within a codon (first, second, and third) evolve at different rates.
Thus, using our knowledge of biological data, we can devise a third approach that further partitions our alignment. 
For this exercise, we will partition sites within the cytB and cox2 gene by codon position.

{\tt \begin{snugshade*}
\begin{lstlisting}
clear()
data_cox2 <- readDiscreteCharacterData("data/primates_cox2.nex")
data_cytb <- readDiscreteCharacterData("data/primates_cytb.nex")
\end{lstlisting}
\end{snugshade*}}

We must now add our codon-partitions to the \cl{data} vector.
The first and second elements in the {\tt data} vector will describe cytB data, and the third and fourth elements will describe cox2 data.
Moreover, the first and third elements will describe the evolutionary process for the first and second codon position sites, while the second and fourth elements describe the process for the third codon position sites alone.

We can create this by calling the helper function \cl{setCodonPartition()}, which is a member function of the data matrix. 
We are assuming that the gene is \textit{in frame}, meaning the first column in your alignment is a first codon position. 
The \cl{setCodonPartition()} function takes a single argument, the position of the alignment you wish to extract. 
It then returns every third column, starting at the index provided as an argument.

Before we can use the use the \cl{setCodonPartition()} function, we must first populate the position in the \cl{data} matrix with some sequences. 
Then we call the member function of \cl{data[1]} to exclude all but the 1$^{st}$ and 2$^{nd}$ positions for cox2.
{\tt \begin{snugshade*}
\begin{lstlisting}
data[1] <- data_cox2
data[1].setCodonPartition( v(1,2) )
\end{lstlisting}
\end{snugshade*}}

Assign the 3$^{rd}$ codon positions for cox2 to \cl{data[2]}:
{\tt \begin{snugshade*}
\begin{lstlisting}
data[2] <- data_cox2
data[2].setCodonPartition( 3 )
\end{lstlisting}
\end{snugshade*}}

Then repeat for cytB, being careful to store the subsetted data to elements 3 and 4:
{\tt \begin{snugshade*}
\begin{lstlisting}
data[3] <- data_cytb
data[3].setCodonPartition( v(1,2) )
data[4] <- data_cytb
data[4].setCodonPartition( 3 )
\end{lstlisting}
\end{snugshade*}}

Now we can query the vector of data matrices to get the size, which is 4:
{\tt \begin{snugshade*}
\begin{lstlisting}
n_parts <- data.size()
\end{lstlisting}
\end{snugshade*}}

Record the number of sites per partition:
{\tt \begin{snugshade*}
\begin{lstlisting}
for (i in 1:n_parts)
    n_sites[i] <- data[i].nchar()
\end{lstlisting}
\end{snugshade*}}

And set the special variables from the data:
{\tt \begin{snugshade*}
\begin{lstlisting}
n_species <- data[1].ntaxa()
names <- data[1].names()
n_branches <- 2 * n_species - 3
\end{lstlisting}
\end{snugshade*}}

Create index variables to populate the move and monitor vectors.
{\tt \begin{snugshade*}
\begin{lstlisting}
mvi <- 1
mni <- 1
\end{lstlisting}
\end{snugshade*}}

Setting up the GTR+$\Gamma$ model is just like in the two-gene analysis, except this time \cl{n\_parts} is equal to 4, so now our vectors of stochastic nodes should all contain nodes for the four pairs of codon-gene partitions.

The script for the model, {\tt RevBayes\_scripts/model\_PS2.Rev}, creates the model parameters for each partition in one large loop.
Like before, we will split the loop into smaller parts to achieve the same end.

First, we will create the GTR rate matrix for partition $i$ by first creating exchangeability rates
{\tt \small \begin{snugshade*}
\begin{lstlisting}
for (i in 1:n_parts) {
  er_prior[i] <- v(1,1,1,1,1,1)
  er[i] ~ dnDirichlet(er_prior[i])
  moves[mvi++] = mvSimplexElementScale(er[i], alpha=10, tune=true, weight=3) 
}
\end{lstlisting}
\end{snugshade*}}

and stationary frequencies

{\tt \small \begin{snugshade*}
\begin{lstlisting}
for (i in 1:n_parts) {
  pi_prior[i] <- v(1,1,1,1)
  pi[i] ~ dnDirichlet(pi_prior[i])
  moves[mvi++] = mvSimplexElementScale(pi[i], alpha=10, tune=true, weight=2)
}
\end{lstlisting}
\end{snugshade*}}

then passing those parameters into a rate matrix function

{\tt \small \begin{snugshade*}
\begin{lstlisting}
for (i in 1:n_parts) {
  Q[i] := fnGTR(er[i],pi[i]) 
}
\end{lstlisting}
\end{snugshade*}}

which states the rate matrix ({\tt Q[i]}) for partition $i$ is determined by the exchangeability rates ({\tt er[i]}) and stationary frequencies ({\tt pi[i]}) also defined for partition $i$.
Following this format, we construct the remainin partition parameters: the $+\Gamma$ mixture model

{\tt \small \begin{snugshade*}
\begin{lstlisting}
for (i in 1:n_parts) {
  alpha_prior_min[i] <- 0.1
  alpha_prior_max[i] <- 50.0
  alpha[i] ~ dnUnif( alpha_prior_min[i], alpha_prior_max[i] )
  norm_gamma_rates[i] := fnDiscretizeGamma( alpha[i], alpha[i], 4, false )
  moves[mvi++] = mvScale(alpha[i], lambda=0.8, tune=true, weight=3.0)
}
\end{lstlisting}
\end{snugshade*}}

the $+I$ invariant sites model

{\tt \small \begin{snugshade*}
\begin{lstlisting}
for (i in 1:n_parts) {
  pinvar[i] ~ dnBeta(1,1)
  moves[mvi++] = mvScale(pinvar[i], lambda=0.1, tune=true, weight=2.0)
  moves[mvi++] = mvSlide(pinvar[i], delta=0.1, tune=true, weight=2.0)
}
\end{lstlisting}
\end{snugshade*}}

and the per-partition molecular clock

{\tt \small \begin{snugshade*}
\begin{lstlisting}
for (i in 1:n_parts) {
  part_rate_mult[i] ~ dnExponential( 1.0 )
  moves[mvi++] = mvScale(part_rate_mult[i], lambda=1.0, tune=true, weight=2.0)
}
\end{lstlisting}
\end{snugshade*}}


Note that applying the per-partition model parameters only depends on the number of partitions, which means the script describing the partitioned model will work so long as you correctly populate the {\tt data} vector.

We are still assuming that the genes share a single topology and branch lengths.
{\tt \begin{snugshade*}
\begin{lstlisting}
speciation ~ dnExponential(10.)
extinction ~ dnExponential(10.)

# rescaling mv on speciation and extinction rates
moves[mvi++] = mvScale(speciation, lambda=1, tune=true, weight=3.0)
moves[mvi++] = mvScale(extinction, lambda=1, tune=true, weight=3.0)

sampling_fraction <- 23. / 270 # 23 out of the ~ 270 primate species

psi ~ dnBDP(lambda=speciation, mu=extinction, rho=sampling_fraction, rootAge=75, names=names)

# moves on the tree
moves[mvi++] = mvNNI(psi, weight=10.0)
moves[mvi++] = mvFNPR(psi, weight=5.0)
moves[mvi++] = mvNodeTimeSlideUniform(psi, weight=10.0)
\end{lstlisting}
\end{snugshade*}}

We must specify a phylogenetic CTMC node for each of our partition models.
{\tt \begin{snugshade*}
\begin{lstlisting}
for (i in 1:n_parts){
  phyloSeq[i] ~ dnPhyloCTMC(tree=psi, Q=Q[i], branchRates=part_rate_mult[i], siteRates=norm_gamma_rates[i], pInv=pinvar[i], type="DNA")
  phyloSeq[i].clamp(data[i])
}

\end{lstlisting}
\end{snugshade*}}

And then wrap up the DAG using the \cl{model()} function:
{\tt \begin{snugshade*}
\begin{lstlisting}
mymodel = model(psi)
\end{lstlisting}
\end{snugshade*}}

Monitor the model parameters, monitoring the tree seperately:

{\tt \begin{snugshade*}
\begin{lstlisting}
monitors[mni++] = mnScreen(er, pi, alpha, norm_gamma_rates, pinvar, speciation, extinction, part_rate_mult, printgen=100)
monitors[mni++] = mnFile(psi, filename="output/PS2.tre", printgen=100)
monitors[mni++] = mnFile(er[1], pi[1], alpha[1], norm_gamma_rates[1], pinvar[1], part_rate_mult[1], er[2], pi[2], alpha[2], norm_gamma_rates[2], pinvar[2], part_rate_mult[2], er[3], pi[3], alpha[3], norm_gamma_rates[3], pinvar[3], part_rate_mult[3], er[4], pi[4], alpha[4], norm_gamma_rates[4], pinvar[4], part_rate_mult[4], speciation, extinction, filename="output/PS2.params.txt",printgen=10)
\end{lstlisting}
\end{snugshade*}}

Set up and run the MCMC analysis:

{\tt \begin{snugshade*}
\begin{lstlisting}
mymcmc = mcmc(mymodel, moves, monitors)
mymcmc.burnin(10000,100)
mymcmc.run(100000) # run for fewer iterations for classroom exercises
\end{lstlisting}
\end{snugshade*}}

And, when MCMC completes, compute the MAP tree:

{\tt \begin{snugshade*}
\begin{lstlisting}
treetrace = readTreeTrace("output/PS2.tre")
treetrace.summarize()
mapTree(treetrace,"output/PS2_map.tre")
\end{lstlisting}
\end{snugshade*}}

\subsection{Exercises}
\begin{itemize}

\item {\bf 1) Reviewing posterior estimates.} Open the {\tt PS2.params.txt} file in Tracer. Remember that partitions 1 and 2 are for cox2, partitions 3 and 4 are for cytB, partitions 1 and 3 are for sites in the first and second codon positions (per gene), and partitions 2 and 4 are for sites in the third and fourth codon positions (per gene).

Aside from the tree topology and divergence times, each partition is modeled to have its own set of parameters.
However, the posterior estimates for some parameters appear quite similar between some pairs of partitions yet different between other pairs of partitions.
For example, {\tt part\_rate\_mult} is the per-partition molecular clock.
This clock is approximately two orders of magnitude faster for partitions 2 and 4 (third codon position sites) than it is for partitions 1 and 3 (non-third codon position sites).

Identify other parameter-partition relationships like this in the posterior.
Under this model, would you consider the gene or the codon site position to hold greater influence over the site's evolutionary mode?

\item {\bf 2) Comparison of MAP trees.} Open the three inferred MAP trees in FigTree.
Check to enable ``Node Labels'', click ``Display'' and select ``posterior'' from the dropdown menu.
Internal nodes now report the probability of the clade appearing in the posterior density of sampled trees.
Do different models yield different tree topologies?
Generally, do complex models provide higher or lower clade support?

\item {\bf 3) Partitioned model selection.}
Bayes factors are computed as the ratio of marginal likelihoods (see the marginal likelihood tutorial (Section XXX) for more details).
Rather than constructing the analysis with an {\tt mcmc} object, marginal likelihood computations rely on output from a {\tt powerPosterior} object.

Copy {\tt model\_PS0.Rev} to {\tt marginal\_PS0.Rev}.
In {\tt marginal\_PS0.Rev}, delete all lines after the {\tt model} function is called, so the MCMC is never run and the MAP tree is never computed.
Additionally, remove the line creating {\tt mnScreen}, which will conflict with output produced by the power posterior analysis.



Instead, configure and run a power posterior analysis

{\tt \begin{snugshade*}
\begin{lstlisting}
pow_p = powerPosterior(mymodel, moves, monitors, "output/model1.out", cats=50)
pow_p.burnin(generations=10000,tuningInterval=1000)
pow_p.run(generations=1000)
\end{lstlisting}
\end{snugshade*}}

then compute the marginal likelihood using the stepping stone sampler

{\tt \begin{snugshade*}
\begin{lstlisting}
ss = steppingStoneSampler(file="output/PS0.out", powerColumnName="power", likelihoodColumnName="likelihood")
ss.marginal()
\end{lstlisting}
\end{snugshade*}}

and again using the path sampler

{\tt \begin{snugshade*}
\begin{lstlisting}
ps = pathSampler(file="PS0.out", powerColumnName="power", likelihoodColumnName="likelihood")
ps.marginal()\end{lstlisting}
\end{snugshade*}}


\item {\bf 4) Customized partition models (advanced).} Create the partitioned model where first, second, and third codon positions are partitioned per gene.
The substitution parameters for each partition are independent {\it except} the first and second codon positions per gene share stationary frequencies.
These parameters would be $\pi_{12}^{(cytB)}$, $\pi_{3}^{(cytB)}$, $\pi_{12}^{(cox2)}$, $\pi_{3}^{(cox2)}$.

The easiest way to accomplish this is to copy {\tt model\_PS2.Rev} to {\tt model\_PS3.Rev} and modify the new file's contents.
Specifically, you will need to adjust how the codon position partitions are defined to yield six (instead of four) partitions.
When looping over partitions, you will need an if-statement to assign stationary frequencies properly.
Take some care when applying monitors and moves to the new model.


\end{itemize}