\section{Overview}


This tutorial focuses on data formats used in \RevBayes.


\subsection*{Requirements}
We assume that you have read and hopefully completed the following tutorials:
\begin{itemize}
\item RB\_Getting\_Started
\end{itemize}



%
%
%
\newpage
\FloatBarrier
\section{Molecular Sequences}

\bigskip
\subsection{Getting Started}


\exs{Download data and output files from: \href{http://revbayes.github.io/tutorials.html}{http://revbayes.github.io/tutorials.html}}
\exs{Open the file \cl{primates\_cytb.nex} in your text editor. This file contains the sequences for the cytochrome B gene sampled from 13 species (Box 1). The elements of the \cl{DATA} block indicate the type of data, number of taxa, and length of the sequences.}


\begin{center}
Box 1: A fragment of the NEXUS file containing the ITS sequences for this exercise. \\
\end{center}
{\tt \scriptsize \begin{framed}
\begin{lstlisting}
#NEXUS 

Begin data;
Dimensions ntax=13 nchar=673;
Format datatype=DNA missing=? gap=-;
Matrix
Trig_excelsa   
TCGAAACCTG...
Fagus_engleriana   
TCGAAACCTG...
Fagus_crenata1   
TCGAAACCTG...
Fagus_japonica2   
TCGAAACCTG...
Fagus_japonica1   
TCGAAACCTG...
Fagus_orientalis   
TCGAAACCTG...
Fagus_sylvatica   
TCGAAACCTG...
Fagus_lucida1   
TCGAAACCTG...
Fagus_lucida2   
TCGAAACCTG...
Fagus_crenata2   
TCGAAACCTG...
Fagus_grandifolia   
TCGAAACCTG...
Fagus_mexicana   
TCGAAACCTG...
Fagus_longipetiolata   
TCGAAACCTG...
	;
End;
\end{lstlisting}
\end{framed}}


\subsection{Reading Molecular Sequences}


{\tt \begin{snugshade*}
\begin{lstlisting}
# show some general information
data
|*
|*
|*

# get some quick and useful attributes of the data
data.ntaxa()
data.names()	
\end{lstlisting}
\end{snugshade*}}



\subsection{Concatenating Sequences}


{\tt \begin{snugshade*}
\begin{lstlisting}
data_atpB <- readDiscreteCharacterData("data/conifer_atpB.nex")
data_atpB <- readDiscreteCharacterData("data/conifer_atpB.nex")
\end{lstlisting}
\end{snugshade*}}

Since the first step in this exercise is to assume a single model for both genes, we need to combine the two datasets.
Concatenate the two data matrices using the \cl{+} operator. This returns a single data matrix with both genes.

{\tt \begin{snugshade*}
\begin{lstlisting}
data <- concatenate(data_ITS, data_rbcL, data_matK)
\end{lstlisting}
\end{snugshade*}}



\subsection{Removing Sites or Genes}


\subsection{Removing Taxa}



\section{Morpholigical Data}



\section{Continuous Trait Data}


\vspace{5cm}
Questions about this tutorial can be directed to: \\\vspace{-10mm}
\begin{itemize}
\item Tracy Heath (email: \href{mailto:tracyh@berkeley.edu}{tracyh@berkeley.edu}) \\\vspace{-8mm}
\item Michael Landis (email: \href{mailto:mlandis@berkeley.edu}{mlandis@berkeley.edu}) \\\vspace{-8mm} 
\item Sebastian H\"{o}hna (email: \href{mailto:sebastian.hoehna@gmail.com}{sebastian.hoehna@gmail.com}) \\\vspace{-8mm}
\item Brian R. Moore (email: \href{mailto:brianmoore@ucdavis.edu}{brianmoore@ucdavis.edu}) \\\vspace{-8mm}
\end{itemize}

\bibliographystyle{mbe}
\bibliography{bib_tex/master_refs}
