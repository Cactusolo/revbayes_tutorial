\section*{Introduction}

A central organizing component of the higher-order architecture of the
genome is chromosome number, and changes in chromosome number have long
been understood to play a fundamental role in evolution.
This tutorial will introduce phylogenetic models of chromosome number evolution,
and demonstrate how to use \RevBayes 
to estimate the rates of chromosome number change and ancestral chromosome numbers.
We will begin by providing an overview of the basic ChromEvol model and 
an example \RevBayes analysis. 
This is followed by a discussion of a number of model extensions 
that enable joint inference of phylogeny
and chromosome numbers, tests for correlated rates of phenotype and chromosome evolution,
and allow for cladogenetic changes in chromosome number.
Next, we will introduce the ChromoSSE model which jointly
estimates diversification rates and chromosome number evolution.
Finally, we will demonstrate how to summarize the results of a \RevBayes
chromosome number analysis and generate plots of ancestral chromosome numbers.

\subsection*{Contents}

The Chromosome Number Evolution tutorial contains several sections:

\begin{itemize}
\item Section \ref{sec:chromo_basic_intro}: Overview of chromosome number evolution models
\item Section \ref{sec:chromo_basic_analysis}: A simple ChromEvol analysis
\item Section \ref{sec:chromo_extensions}: Basic extensions of the model
\item Section \ref{sec:chromo_sse_intro}: Overview of the ChromoSSE model
\item Section \ref{sec:chromo_sse_analysis}: A simple ChromoSSE analysis
\end{itemize}

\subsection*{Recommended tutorials}

This tutorial assumes the reader is familiar with the content covered in the following \RevBayes tutorials:

\begin{itemize}
\item {\bf {\tt \large Rev} Basics}
\item {\bf Molecular Models of Character Evolution}
\item {\bf Discrete Morphology Models of Character Evolution.}
\item {\bf Running and Diagnosing an MCMC Analysis}
\end{itemize}

\newpage
