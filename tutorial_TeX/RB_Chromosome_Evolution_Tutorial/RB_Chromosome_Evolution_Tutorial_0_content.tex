\section*{Introduction}

A central organizing component of the higher-order architecture of the
genome is chromosome number, and changes in chromosome number have long
been understood to play a fundamental role in evolution.
This tutorial will introduce phylogenetic models of chromosome number evolution,
and demonstrate how to use \RevBayes 
to estimate the rates of chromosome number change and ancestral chromosome numbers.
We will also show how to use the \RevGadgets R package to make plots of ancestral
chromosome number estimates and stochastic character maps of chromosome evolution.

We will begin by providing an overview of the basic ChromEvol model \citep{mayrose2010probabilistic} and 
an example \RevBayes analysis. 
This is followed by a discussion of a number of model extensions 
that enable joint inference of phylogeny
and chromosome numbers, 
tests for correlated rates of phenotype and chromosome evolution \citep[the BiChroM model;][]{zenil2017testing},
and incorporating cladogenetic changes in chromosome number.
Next, we will introduce the ChromoSSE model \citep{freyman2016cladogenetic} which jointly
estimates diversification rates and chromosome number evolution.
We also briefly discuss testing hypotheses of chromosome evolution by
comparing different models using reversible-jump MCMC and Bayes factors.

If you use \RevBayes for chromosome evolution analyses, please cite the original 
papers that describe the chromosome evolution models as well as \citet{freyman2016cladogenetic}
which describes in detail the \RevBayes implementation of these models.

\subsection*{Contents}

The Chromosome Number Evolution tutorial contains several sections:

\begin{itemize}
\item Section \ref{sec:chromo_basic_intro}: Overview of chromosome number evolution models
\item Section \ref{sec:chromo_basic_analysis}: A simple ChromEvol analysis
\item Section \ref{sec:chromo_extensions}: Basic extensions of the model
    \begin{itemize}
    \item \ref{subsect:root_freq}: Improved root frequences
    \item \ref{subsect:stoch_mapping}: Stochastic character mapping of chromosome evolution
    \item \ref{subsect:joint_estimation}: Joint estimation of phylogeny and chromosome evolution
    \item \ref{subsect:bichrom}: Associating chromosome evolution with phenotype: BiChroM
    \item \ref{subsect:clado_simple}: Incorporating cladogenetic and anagenetic chromosome changes
    \end{itemize}
\item Section \ref{sec:chromosse_intro}: Overview of the ChromoSSE model
\item Section \ref{sec:chromosse_analysis}: A simple ChromoSSE analysis
\end{itemize}

\subsection*{Example scripts and data}

The data and the full scripts used for all the examples can be found on the \RevBayes website:

\begin{itemize}
\item \href{http://rawgit.com/revbayes/revbayes_tutorial/master/RB_Chromosome_Evolution_Tutorial/scripts.zip}{\cl{scripts.zip}}
\item \href{http://rawgit.com/revbayes/revbayes_tutorial/master/RB_Chromosome_Evolution_Tutorial/data.zip}{\cl{data.zip}}
\end{itemize}

\subsection*{Recommended tutorials}

This tutorial assumes the reader is familiar with the content covered in the following \RevBayes tutorials:

\begin{itemize}
\item {\bf {\tt \large Rev} Basics}
\item {\bf Molecular Models of Character Evolution}
\item {\bf Discrete Morphology Models of Character Evolution}
\item {\bf Running and Diagnosing an MCMC Analysis}
\end{itemize}

\newpage
