\section*{Introduction}

How did species come to live where they're found today?
To answer this, we can leverage phylogenetic and geological information to model species distributions as the outcome of biogeographic processes.
These natural processes require some additional considerations, such as how ranges are inherited following speciation events, how geological events might influence dispersal rates, and what factors affect rates of dispersal and extirpation.
The major challenge of modeling range evolution is how to translate these natural processes into stochastic processes that remain tractable for inference.
This tutorial provides a brief background in some of these models, then describes how to perform Bayesian inference of historical biogeography using \RevBayes.

%{\bf 1. Introduction}
%{\bf 1. Range evolution}
%
%\quad a) Dispersal-Extinction-Cladogenesis
%
%\quad b) Setting up the model
%
%{\bf II. Advanced features (under development)}
%
%\quad a) Epoch model
%
%\quad b) GIS layers
%
%\quad c) Ancestral range estimation
%
%{\bf III. Data augmented biogeography}
%
%\quad a) Data augmentation, non-independent character evolution
%
%\quad b) Setting up the model
%
%\quad c) Phylowood animation

%\noindent \\ \impmark These little arrows indicate lines containing key information to progress through the lab. The rest of the text gives context for why we're taking these steps or what to make of results.
%
%\subsection{Handy links for this lab}

%\begin{tabular}{ll}
%RevBayes & \url{https://github.com/revbayes/revbayes} \\
%Lab zip file & \href{https://github.com/revbayes/revbayes/raw/development/tutorials/RB\_Biogeography\_tutorial/RB\_biogeo\_files.zip}{{\tt https://github.com/revbayes/revbayes/.../RB\_biogeo\_files.zip}} \\
%Phylowood software & \url{http://mlandis.github.io/phylowood} \\
%Phylowood manual & \url{https://github.com/mlandis/phylowood/wiki} \\
%Tracer & \url{http://tree.bio.ed.ac.uk/software/tracer}
%
%\end{tabular}

%\subsection{Setting up your workspace}
%
%The practical part of the lab will analyze a small dataset of 19 taxa distributed over 4 biogeographic areas.
%Parts of the lab will require entering terminal commands, which will assume you are using the Unix shell \texttt{bash}.
%Just ask for help if the commands don't seem to work.
%
%\noindent \\ \impmark Portions of this lab require Python is installed. No packages are needed.
%
%\noindent \\ \impmark This tutorial will assume you have successfully installed RevBayes and can be called from your current working directory.
%
%\noindent \\ \impmark Download and unzip the lab zip file, {\tt RB\_biogeo\_files.zip}.

%%%%%%%%%%%%%%%%
%%%%%%%%%%%%%%%%
