\section{Ancestral state estimation} \label{sec:dm_matrix}

Discrete morphological models are not only useful for tree estimation, as was done in Section \ref{sec:dm_overview}, but also for ancestral state estimation.
The central problem in statistical phylogenetics concerns {\it marginalizing} over all unobserved character histories that evolved along the branches of a given phylogenetic tree.
This allows one to compute the likelihood of observing a character matrix given they evolved according to an underlying tree.
While marginalizing over those histories, one might also seek to produce samples of ancestral states that are consistent with the tip state distribution.

\begin{align*}
P( X_\text{tip} ) & = \sum_{x \in X_\text{internal}} P( X_\text{tip}, X_\text{internal} )
\end{align*}

We want the joint probabilities of internal node states conditioned on the tips
\begin{align*}
P( X^{(\text{internal})} \mid X^{(\text{tip})} ) & = \sum_{x \in X_\text{internal}} P( X_\text{tip}, X_\text{internal} )
\end{align*}

or the marginal probabilities conditioned on the tips

\begin{align*}
P( X_i^{(\text{internal})} \mid X^{(\text{tip})} ) & = \sum_{x \in X_\text{internal}} P( X_\text{tip}, X_\text{internal} )
\end{align*}


Ancestral state estimation is the procedure of sampling ancestral states that are consistent with the observed distribution of tip states.
There are several strategies for doing so, each with their own advantages and disadvantages.
One can compute the {\it exact probability} of any given ancestral node having an ancestral state value; or one can {\it sample} states from that distribution in proportion to the exact probability.
One can produce {\it marginal} ancestral state estimates that report how often any given node is in a state without the context of the ancestral states of the remaining nodes; or one can produce {\it joint} ancestral state estimates that report the probability of particular combinations of sequences of evolutionary histories, running from the root to the tips of the phylogeny.
While it is straightforward to produce exact probabilities for marginal histories, sampling is generally required to produce joint histories.

\subsection{Ancestral state monitors}

We focus on joint samples in this tutorial and in \RevBayes because they yield most direct representation of possible evolutionary scenarios.
The completed exercises from Section \ref{sec:dm_overview} made use of a monitor names {\tt mnJointConditionalAncestralState}, which was created with the command

{\tt \begin{snugshade*}
\begin{lstlisting}
monitors[mni++] = mnJointConditionalAncestralState(tree=phylogeny,
                                                   ctmc=phyMorpho,
                                                   filename="output/simple.states.txt",
                                                   type="Standard",
                                                   printgen=10,
                                                   withTips=false)
\end{lstlisting}
\end{snugshade*}}

The core arguments this monitor needs are a tree object ({\tt tree=phylogeny}), the phylogenetic model ({\tt ctmc=phyMorpho}), an output filename ({\tt filename="output/simple.states.txt"}), the data type for the characters ({\tt type="Standard"}), and the sampling frequency ({\tt printgen=10}).
The final argument, {\tt withTips=false}, indicates that we do not wish to record the starting state for each branch, since it will always be identical to the ending state of the parental branch (but this is not necessarily so for models with cladogenesis).

The monitor will produce a joint sample of ancestral states (conditional on the tip states) every 10 iterations, storing the samples to the file {\tt "output/simple.states.txt"}.
Viewing the states file, we see

{\tt \begin{snugshade*}
\begin{lstlisting}
Iteration	end_1	end_2	end_3	end_4
0	?,0,?,?,?,?,?,?,?,?,?,?,?,?,?,?,0,0,0, ...
10	?,0,?,?,?,?,?,?,?,?,?,?,?,?,?,?,0,0,0, ...
20	?,0,?,?,?,?,?,?,?,?,?,?,?,?,?,?,0,0,0, ...
30	?,0,?,?,?,?,?,?,?,?,?,?,?,?,?,?,0,0,0, ...
40	?,0,?,?,?,?,?,?,?,?,?,?,?,?,?,?,0,0,0, ...
50	?,0,?,?,?,?,?,?,?,?,?,?,?,?,?,?,0,0,0, ...
...
\end{lstlisting}
\end{snugshade*}}

The first column is the MCMC iteration, and the remaining columns represent the states sampled at end of the branch for an indexed node.
When the phylogeny is a random variable, the index is used to map an ancestral state estimate to the node in any given tree sample from the analysis.
The vector of comma-delimited ancestral state estimates is reported for each column value.
In the example above, we see that the node indexed 1 (a tip node) always produces the observed character matrix [ future versions will sample ambiguous tip states, e.g. {\tt ?} ].
The first $1 \ldots N$ index values are assigned to tip nodes in a tree with $N$ tips.
We observe the uncertainty in our ancestral state estimates by viewing the final column

{\tt \begin{snugshade*}
\begin{lstlisting}
[ column headers are far to the left ]
... 1,0,1,0,1,0,1,1,0,0,0,1,1,1
... 1,1,1,1,1,0,0,1,0,0,1,1,1,0
... 1,1,1,0,0,0,0,0,0,0,0,0,1,0
... 1,1,0,1,1,0,1,0,0,1,0,0,1,0
... 1,1,0,1,1,0,1,0,1,0,0,0,1,0
... 1,1,0,1,1,0,1,0,1,1,0,0,1,0
...
\end{lstlisting}
\end{snugshade*}}

\subsection{Ancestral state figures}

Now that we have our posterior distribution of ancestral states and of phylogenetic trees, we want to summarize those results into something simpler.
This section will aim to produce a pdf containing figures for the ancestral state estimates mapped onto a consensus tree for all 62 characters.

\subsection{Output processing}

To accomplish this, we will rely on a combination of \RevBayes, {\tt R}, and {\tt bash} scripts.
The contents of the following scripts demonstrate how one might integrate \RevBayes into a pipeline to process output using a variety of tools.
Precomputed results are found in {\tt example\_output}.
[ Unfortunately, this exact pipeline probably does not work for Windows users at this time. ]

To generate the figure {\tt output/mk\_simple.char\_1\_to\_5.pdf} containing ancestral state estimates for characters 1 through 5, use the following command

{\tt \begin{snugshade*}
\begin{lstlisting}
./scripts/make_anc_trees.sh output/mk_simple 1 5
\end{lstlisting}
\end{snugshade*}}

This shell script first generates ancestral state trees using the analysis files prefixed with {\tt "output/mk\_simple"} for characters {\tt 1} through {\tt 5} by piping those variables into the \RevBayes script {\tt scripts/make\_anc\_states.Rev}.
The {\tt make\_anc\_states.Rev} script produces a series of tree files annotated with the ancestral state estimates and probabilities named e.g. {\tt mk\_simple.char\_1.ase.tre}.
Next, the shell script calls {\tt Rscript} to execute {\tt scripts/plot\_anc\_state.R} using the tree filenames as arguments.
The {\tt plot\_anc\_states.R} file produces a tree figure for each file using the {\tt R} package, {\tt RevGadgets}.
Finally, the shell script combines the resulting figures into a single concatenated pdf using the tool {\tt gs}.

\subsection{Results}

Knowing what we do about the various models, how might we expect the ancestral state estimates to differ?
For one, the {\tt simple} model does not condition on ascertainment bias towards variable characters.
This leads to overestimated branch lengths when compared to the {\tt mk\_simple} model.

We might expect that {\tt mk\_simple}, because it forces all characters to evolve under a single symmetric model of evolution, would have more ambiguity in ancestral state estimates.
In contrast {\tt mk\_discretized} and {\tt mk\_mixture} allow characters to evolve with different tendencies towards 0- or 1-valued states, which might increase the certainty in ancestral state estimates.


\newpage
