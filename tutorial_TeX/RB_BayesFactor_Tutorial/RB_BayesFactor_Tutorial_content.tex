\section{Overview}


This tutorial demonstrates some general principles of Bayesian model comparison, which is based on estimating the marginal likelihood of competing models and then comparing their relative fit to the data using Bayes factors.
We consider the specific case of calculating Bayes factors to select among different substitution models.
% and different partition configurations of aligned DNA sequences. 

\subsection{Requirements}
We assume that you have read and hopefully completed the following tutorials:
\begin{itemize}
\item RB\_Getting\_Started
\item RB\_Data\_Tutorial
\item RB\_CTMC\_Tutorial
\end{itemize}
This means that we will assume that you know how to execute and load data into \RevBayes, are familiar with some basic commands, and know how to perform an analysis of a single-gene dataset (assuming an unconstrained/unrooted tree).



%%%%%%%%
%%   Data   %%
%%%%%%%%
\section{Data and files}

We provide several data files that we will use in this tutorial.
Of course, you may want to use your own dataset instead.
In the \cl{data} folder, you will find the following files
\begin{itemize}
\item
\cl{primates\_cytb.nex}: Alignment of the \textit{cytochrome b} subunit from 23 primates representing 14 of the 16 families (\textit{Indriidae} and \textit{Callitrichidae} are missing).
\item
\cl{primates\_16s.nex}: Alignment of the \textit{16s ribosomal RNA} gene from the same 23 primates species.
\item
\cl{primates\_cox2.nex}: Alignment of the \textit{COX-2} gene from the same 23 primates species.
\end{itemize}



\section{Introduction}

For most sequence alignments, several (possibly many) substitution models of varying complexity are plausible {\it a priori}.
We therefore need a way to objectively identify the model that balances estimation bias and inflated error variance associated with under- and over-parameterized models, respectively.
Increasingly, model selection is based on \textit{Bayes factors} \citep[{\it e.g.},][]{Suchard2001,Lartillot2006,Xie2011,Baele2012,Baele2013}, which involves first calculating the marginal likelihood of each candidate model and then comparing the ratio of the marginal likelihoods for the set of candidate models.

 
Given two models, $M_0$ and $M_1$, the Bayes-factor comparison assessing the relative fit of each model to the data, $BF(M_0,M_1)$, is:
$$BF(M_0,M_1) = \frac{\mbox{posterior odds}}{\mbox{prior odds}}.$$
The posterior odds is the posterior probability of $M_0$ given the data, $\mathbf X$, divided by the posterior odds of $M_1$ given the data:
$$\mbox{posterior odds} = \frac{\mathbb{P}(M_0 \mid \mathbf X)}{\mathbb{P}(M_1 \mid \mathbf X)},$$
and the prior odds is the prior probability of $M_0$ divided by the prior probability of $M_1$:
$$\mbox{prior odds} = \frac{\mathbb{P}(M_0)}{\mathbb{P}(M_1)}.$$
Thus, the Bayes factor measures the degree to which the data alter our belief regarding the support for $M_0$ relative to $M_1$ \citep{Lavine1999}:
\begin{align}\label{BFeq1}
BF(M_0,M_1) = \frac{\mathbb{P}(M_0 \mid \mathbf X, \theta_0)}{\mathbb{P}(M_1 \mid \mathbf X, \theta_1)} \div \frac{\mathbb{P}(M_0)}{\mathbb{P}(M_1)}. 
\end{align}
Note that interpreting Bayes factors involves some subjectivity.
That is, it is up to \textsl{you} to decide the degree of your belief in $M_0$ relative to $M_1$. 
Despite the absence of an absolutely objective model-selection threshold, we can refer to the scale \citep[outlined by][]{Jeffreys1961} that provides a ``rule-of-thumb'' for interpreting these measures (Table \ref{bftable}).
\begin{table}[h]
\centering
\caption{\small The scale for interpreting Bayes factors by Harold \citet{Jeffreys1961}.} 
\label{bftable}
\begin{tabular}{l c c c}
\hline
\multicolumn{1}{r}{{Strength of evidence}} & \multicolumn{1}{l}{\textbf{$BF(M_0, M_1)$}} & \multicolumn{1}{l}{\textbf{log($BF(M_0, M_1)$)}} &  \multicolumn{1}{l}{\textbf{$\text{log}_{10}$($BF(M_0, M_1)$)}}\\ 
\hline
Negative (supports $M_1$) & $<1$ & $<0$ & $<0$\\
Barely worth mentioning & $1$ to $3.2$ & $0$ to $1.16$ & $0$ to $0.5$\\
Substantial & $3.2$ to $10$ & $1.16$ to $2.3$ & $0.5$ to $1$ \\
Strong & $10$ to $100$ & $2.3$ to $4.6$ & $1$ to $2$ \\
Decisive& $>100$ & $>4.6$ & $>2$ \\
\hline
\multicolumn{3}{l}{{\scriptsize{For a detailed description of Bayes factors see \citet{Kass1995}}}} 
\end{tabular}
\end{table}


Unfortunately, it is generally not possible to directly calculate the posterior odds to prior odds ratio. 
However, we can further define the posterior odds ratio as:
\begin{align*}
\frac{\mathbb{P}(M_0 \mid \mathbf X)}{\mathbb{P}(M_1 \mid \mathbf X)} = \frac{\mathbb{P}(M_0)}{\mathbb{P}(M_1)} \frac{\mathbb{P}(\mathbf X \mid M_0)}{\mathbb{P}(\mathbf X \mid M_1)},
\end{align*}
where $\mathbb{P}(\mathbf X \mid M_i)$ is the \textit{marginal likelihood} of the data (this may be familiar to you as the denominator of Bayes Theorem, which is variously referred to as the \textit{model evidence} or \textit{integrated likelihood}).
Formally, the marginal likelihood is the probability of the observed data ($\mathbf X$) under a given model ($M_i$) that is averaged over all possible values of the parameters of the model ($\theta_i$) with respect to the prior density on $\theta_i$
\begin{align}\label{margeLike}
\mathbb{P}(\mathbf X \mid M_i) = \int \mathbb{P}(\mathbf X \mid \theta_i) \mathbb{P}(\theta_i)dt.
\end{align}
This makes it clear that more complex (parameter-rich) models are penalized by virtue of the associated prior: each additional parameter entails integration of the likelihood over the corresponding prior density.  
If you refer back to equation \ref{BFeq1}, you can see that, with very little algebra, the ratio of marginal likelihoods is equal to the Bayes factor:
\begin{align}\label{bfFormula}
BF(M_0,M_1) = \frac{\mathbb{P}(\mathbf X \mid M_0)}{\mathbb{P}(\mathbf X \mid M_1)} = \frac{\mathbb{P}(M_0 \mid \mathbf X, \theta_0)}{\mathbb{P}(M_1 \mid \mathbf X, \theta_1)} \div \frac{\mathbb{P}(M_0)}{\mathbb{P}(M_1)}. 
\end{align}
Therefore, we can perform a Bayes factor comparison of two models by calculating the marginal likelihood for each one. % Simple as pie, right?
Alas, exact solutions for calculating marginal likelihoods are not known for phylogenetic models (see equation \ref{margeLike}), thus we must resort to numerical integration methods to estimate or approximate these values. 
In this exercise, we will estimate the marginal likelihood for each partition scheme
using both the stepping-stone \citep{Xie2011,Fan2011} and path sampling estimators \citep{Lartillot2006, Baele2012}. 



\bigskip
\subsection{Substitution Models}\label{secUnif} 

The models we use here are equivalent to the models described in the previous exercise on substitution models (continuous time Markov models).
To specify the model please consult the previous exercise. Specifically, you will need to specify the following substitution models:
\begin{itemize}
\item Jukes-Cantor (JC) substitution model \citep[][]{Jukes1969}
\item Hasegawa-Kishino-Yano (HKY) substitution model \citep[][]{Hasegawa1985}
\item General-Time-Reversible (GTR) substitution model \citep[][]{Tavare1986}
\item Gamma (+G) model for among-site rate variation \citep[][]{yang94a}
\item Invariable-sites (+I) model \citep[][]{Hasegawa1985}
\end{itemize}


\bigskip
\subsection{Estimating the Marginal Likelihood}

We will estimate the marginal likelihood of a given model using a `stepping-stone' (or `path-sampling') algorithm.
These algorithms are similar to the familiar MCMC algorithms, which are intended to sample from (and estimate) the joint posterior probability of the model parameters.
Stepping-stone algorithms are like a series of MCMC simulations that iteratively sample from a specified number of discrete steps between the posterior and the prior probability distributions.
The basic idea is to estimate the probability of the data for all points between the posterior and the prior---effectively summing the probability of the data over the prior probability of the parameters to estimate the marginal likelihood. 
Technically, the steps correspond to a series of \cl{powerPosteriors()}: a series of numbers between 1 and 0 that are iteratively applied to the posterior.
When the posterior probability is raised to the power of 1 (typically the first stepping stone), samples are drawn from the (untransformed) posterior.
By contrast, when the posterior probability is raised to the power of 0 (typically the last stepping stone), samples are drawn from the prior.
To perform a stepping-stone simulation, we need to specify (1) the \emph{number} of stepping stones (power posteriors) that we will use to traverse the path between the posterior and the prior (e.g., we specify 50 or 100 stones), (2) the \emph{spacing} of the stones between the posterior and prior (e.g., we may specify that the stones are distributed according to a beta distribution), (3) the number of samples to (and thinning) of samples to be drawn from each stepping stone, and (4) the direction we will take (i.e., from the posterior to the prior or vice versa).

%With a fully specified model, we can set up the \cl{powerPosterior()} analysis to create a file of `powers' and likelihoods from which we can estimate the marginal likelihood using stepping-stone or path sampling. 
This method computes a vector of powers from a beta distribution, then executes an MCMC run for each power step while raising the likelihood to that power. In this implementation, the vector of powers starts with 1, sampling the likelihood close to the posterior and incrementally sampling closer and closer to the prior as the power decreases. 



%\textbf{\textit{Clear Workspace and Load the Data and Model}}

Just to be safe, it is better to clear the workspace (if you did not just restart \RevBayes)
{\tt \begin{snugshade*}
\begin{lstlisting}
clear()
\end{lstlisting}
\end{snugshade*}}

Now set up the model as in the previous exercise. You should start with the simple Jukes-Cantor substitution model. 
Setting up the model requires:
\begin{enumerate}
\item Loading the data and retrieving useful variables about it (\EG number of sequences and taxon names).
\item Specifying the instantaneous-rate matrix of the substitution model.
\item Specifying the tree model including branch-length variables.
\item Creating a random variable for the sequences that evolved under the \cl{PhyloCTMC}.
\item Clamping the data.
\item Creating a model object.
\item Specifying the moves for parameter updates.
\end{enumerate}

The following procedure for estimating marginal likelihoods is valid for any model in \RevBayes.
You will need to repeat this later for other models.
First, we create the variable containing the power-posterior analysis. 
This requires that we provide a model and vector of moves, as well as an output file name. 
The \cl{cats} argument sets the number of stepping stones.
{\tt \begin{snugshade*}
\begin{lstlisting}
pow_p = powerPosterior(mymodel, moves, "model1.out", cats=50) 
\end{lstlisting}
\end{snugshade*}}

We can start the power-posterior analysis by first burning in the chain and and discarding the first 10000 states.  
This will help ensure that analysis starts from a region of high posterior probability, rather than from some random point.
{\tt \begin{snugshade*}
\begin{lstlisting}
pow_p.burnin(generations=10000,tuningInterval=1000)
\end{lstlisting}
\end{snugshade*}}

Now execute the run with the \cl{.run()} function:
{\tt \begin{snugshade*}
\begin{lstlisting}
pow_p.run(generations=1000)  
\end{lstlisting}
\end{snugshade*}}

Once the power posteriors have been saved to file, create a stepping stone sampler. 
This function can read any file of power posteriors and compute the marginal likelihood using stepping-stone sampling. 
{\tt \small \begin{snugshade*}
\begin{lstlisting}
ss <- steppingStoneSampler(file="model1.out", powerColumnName="power", likelihoodColumnName="likelihood")
\end{lstlisting}
\end{snugshade*}}

These commands will execute a stepping-stone simulation with 50 stepping stones, sampling 1000 states from each step. 
Compute the marginal likelihood under stepping-stone sampling using the member function \cl{marginal()} of the \cl{ss} variable and record the value in Table \ref{tab:ml_cytb}.
{\tt \begin{snugshade*}
\begin{lstlisting}
ss.marginal() 
\end{lstlisting}
\end{snugshade*}}

Path sampling is an alternative to stepping-stone sampling and also takes the same power posteriors as input. 
{\tt \small \begin{snugshade*}
\begin{lstlisting}
ps = pathSampler(file="model1.out", powerColumnName="power", likelihoodColumnName="likelihood")
\end{lstlisting}
\end{snugshade*}}

Compute the marginal likelihood under stepping-stone sampling using the member function \cl{marginal()} of the \cl{ps} variable and record the value in Table \ref{tab:ml_cytb}.
{\tt \begin{snugshade*}
\begin{lstlisting}
ps.marginal() 
\end{lstlisting}
\end{snugshade*}}

\noindent \\ \impmark As an example we provide the file \textbf{RevBayes\_scripts/marginalLikelihood\_JukesCantor.Rev}.


\subsection{Exercises}

\begin{itemize}
\item Compute the marginal likelihoods of the \textit{cytb} alignment for the following substitution models:
\begin{itemize}
\item Jukes-Cantor (JC) substitution model
\item Hasegawa-Kishino-Yano (HKY) substitution model
\item General-Time-Reversible (GTR) substitution model
\item GTR with gamma distributed-rate model (GTR+G)
\item GTR with invariable-sites model  (GTR+I)
\item GTR+I+G model
\end{itemize}
\item Enter the marginal likelihood estimate for each model in the corresponding cell of Table~\ref{tab:ml_cytb}.
\item Repeat the above marginal likelihood analyses for the \textit{mt-COX2 gene} and enter results in Table~\ref{tab:ml_cox2}.
\item Which is the best fitting substitution model?
\end{itemize}

\begin{Form}
\begin{table}[h]
\centering
\caption{\small Estimated marginal likelihoods for different substitution models for the cytb alignment$^*$.}
\begin{tabular}{l c c c c}
\hline
\multicolumn{1}{l}{\textbf{ }} &\multicolumn{1}{r}{\textbf{ }} & \multicolumn{3}{c}{\textbf{Marginal lnL estimates}} \\ 
\cline{3-5}
\multicolumn{1}{l}{\textbf{Substitution Model}} & \multicolumn{1}{r}{\hspace{3mm}} & \multicolumn{1}{c}{\textit{Stepping-stone}} & \multicolumn{1}{r}{\hspace{3mm}} & \multicolumn{1}{c}{\textit{Path sampling}} \\ 
\hline
JC ($M_1$) & \hspace{15mm} & \TextField[name=gene1_m11,backgroundcolor={.85 .85 .85},color={1 0 0},height=4ex]{}  & \hspace{15mm} & \TextField[name=gene1_m12,backgroundcolor={.85 .85 .85},color={0 0 1},height=4ex]{} \\
\hline
HKY ($M_2$) & \hspace{3mm} &\TextField[name=gene1_m21,backgroundcolor={.85 .85 .85},color={1 0 0},height=4ex]{}   & \hspace{3mm} & \TextField[name=gene1_m22,backgroundcolor={.85 .85 .85},color={0 0 1},height=4ex]{} \\
\hline
GTR ($M_3$) & \hspace{3mm} &\TextField[name=gene1_m31,backgroundcolor={.85 .85 .85},color={1 0 0},height=4ex]{}   & \hspace{3mm} & \TextField[name=gene1_m32,backgroundcolor={.85 .85 .85},color={0 0 1},height=4ex]{} \\
\hline
GTR+$\Gamma$ ($M_4$) & \hspace{3mm} & \TextField[name=gene1_m41,backgroundcolor={.85 .85 .85},color={1 0 0},height=4ex]{} & \hspace{3mm} & \TextField[name=gene1_m42,backgroundcolor={.85 .85 .85},color={0 0 1},height=4ex]{} \\
\hline
GTR+I ($M_5$) & \hspace{3mm} & \TextField[name=gene1_m51,backgroundcolor={.85 .85 .85},color={1 0 0},height=4ex]{} & \hspace{3mm} & \TextField[name=gene1_m52,backgroundcolor={.85 .85 .85},color={0 0 1},height=4ex]{} \\
\hline
GTR+$\Gamma$+I ($M_6$) & \hspace{3mm} & \TextField[name=gene1_m61,backgroundcolor={.85 .85 .85},color={1 0 0},height=4ex]{} & \hspace{3mm} & \TextField[name=gene1_m62,backgroundcolor={.85 .85 .85},color={0 0 1},height=4ex]{} \\
\hline
Any other model ($M_7$) & \hspace{3mm} & \TextField[name=gene1_m71,backgroundcolor={.85 .85 .85},color={1 0 0},height=4ex]{} & \hspace{3mm} & \TextField[name=gene1_m72,backgroundcolor={.85 .85 .85},color={0 0 1},height=4ex]{} \\
\hline
Any other model ($M_8$) & \hspace{3mm} & \TextField[name=gene1_m81,backgroundcolor={.85 .85 .85},color={1 0 0},height=4ex]{} & \hspace{3mm} & \TextField[name=gene1_m82,backgroundcolor={.85 .85 .85},color={0 0 1},height=4ex]{} \\
\hline
Any other model ($M_9$) & \hspace{3mm} & \TextField[name=gene1_m91,backgroundcolor={.85 .85 .85},color={1 0 0},height=4ex]{} & \hspace{3mm} & \TextField[name=gene1_m92,backgroundcolor={.85 .85 .85},color={0 0 1},height=4ex]{} \\
\hline
{\footnotesize{$^*$you can edit this table}}\\
\end{tabular}
\label{tab:ml_cytb}
\end{table}
\end{Form}


\begin{Form}
\begin{table}[h]
\centering
\caption{\small Estimated marginal likelihoods for different substitution models of the cox2 gene$^*$.}
\begin{tabular}{l c c c c}
\hline
\multicolumn{1}{l}{\textbf{ }} &\multicolumn{1}{r}{\textbf{ }} & \multicolumn{3}{c}{\textbf{Marginal lnL estimates}} \\ 
\cline{3-5}
\multicolumn{1}{l}{\textbf{Substitution Model}} & \multicolumn{1}{r}{\hspace{3mm}} & \multicolumn{1}{c}{\textit{Stepping-stone}} & \multicolumn{1}{r}{\hspace{3mm}} & \multicolumn{1}{c}{\textit{Path sampling}} \\ 
\hline
JC ($M_1$) & \hspace{15mm} & \TextField[name=gene2_m11,backgroundcolor={.85 .85 .85},color={1 0 0},height=4ex]{}  & \hspace{15mm} & \TextField[name=gene2_m12,backgroundcolor={.85 .85 .85},color={0 0 1},height=4ex]{} \\
\hline
HKY ($M_2$) & \hspace{3mm} &\TextField[name=gene2_m21,backgroundcolor={.85 .85 .85},color={1 0 0},height=4ex]{}   & \hspace{3mm} & \TextField[name=gene2_m22,backgroundcolor={.85 .85 .85},color={0 0 1},height=4ex]{} \\
\hline
GTR ($M_3$) & \hspace{3mm} &\TextField[name=gene2_m31,backgroundcolor={.85 .85 .85},color={1 0 0},height=4ex]{}   & \hspace{3mm} & \TextField[name=gene2_m32,backgroundcolor={.85 .85 .85},color={0 0 1},height=4ex]{} \\
\hline
GTR+$\Gamma$ ($M_4$) & \hspace{3mm} & \TextField[name=gene2_m41,backgroundcolor={.85 .85 .85},color={1 0 0},height=4ex]{} & \hspace{3mm} & \TextField[name=gene2_m42,backgroundcolor={.85 .85 .85},color={0 0 1},height=4ex]{} \\
\hline
GTR+I ($M_5$) & \hspace{3mm} & \TextField[name=gene2_m51,backgroundcolor={.85 .85 .85},color={1 0 0},height=4ex]{} & \hspace{3mm} & \TextField[name=gene2_m52,backgroundcolor={.85 .85 .85},color={0 0 1},height=4ex]{} \\
\hline
GTR+$\Gamma$+I ($M_6$) & \hspace{3mm} & \TextField[name=gene2_m61,backgroundcolor={.85 .85 .85},color={1 0 0},height=4ex]{} & \hspace{3mm} & \TextField[name=gene2_m62,backgroundcolor={.85 .85 .85},color={0 0 1},height=4ex]{} \\
\hline
Any other model ($M_7$) & \hspace{3mm} & \TextField[name=gene2_m71,backgroundcolor={.85 .85 .85},color={1 0 0},height=4ex]{} & \hspace{3mm} & \TextField[name=gene2_m72,backgroundcolor={.85 .85 .85},color={0 0 1},height=4ex]{} \\
\hline
Any other model ($M_8$) & \hspace{3mm} & \TextField[name=gene2_m81,backgroundcolor={.85 .85 .85},color={1 0 0},height=4ex]{} & \hspace{3mm} & \TextField[name=gene2_m82,backgroundcolor={.85 .85 .85},color={0 0 1},height=4ex]{} \\
\hline
Any other model ($M_9$) & \hspace{3mm} & \TextField[name=gene2_m91,backgroundcolor={.85 .85 .85},color={1 0 0},height=4ex]{} & \hspace{3mm} & \TextField[name=gene2_m92,backgroundcolor={.85 .85 .85},color={0 0 1},height=4ex]{} \\
\hline
{\footnotesize{$^*$you can edit this table}}\\
\end{tabular}
\label{tab:ml_cox2}
\end{table}
\end{Form}





\FloatBarrier
\section{Compute Bayes Factors and Select Model}


Now that we have estimates of the marginal likelihood for each of our the candidate substitution models, we can evaluate their relative fit to the datasets using Bayes factors.
Phylogenetic programs log-transform the likelihood values to avoid \href{http://en.wikipedia.org/wiki/Arithmetic_underflow}{underflow}: multiplying likelihoods (numbers $< 1$) generates numbers that are too small to be held in computer memory.
Accordingly, we need to use a different form of equation \ref{bfFormula} to calculate the ln-Bayes factor (we will denote this value $\mathcal{K}$):
\begin{align}\label{LNbfFormula}
\mathcal{K}=\ln[BF(M_0,M_1)] = \ln[\mathbb{P}(\mathbf X \mid M_0)]-\ln[\mathbb{P}(\mathbf X \mid M_1)],
\end{align}
where $\ln[\mathbb{P}(\mathbf X \mid M_0)]$ is the \textit{marginal lnL} estimate for model $M_0$. 
The value resulting from equation \ref{LNbfFormula} can be converted to a raw Bayes factor by simply taking the exponent of $\cal{K}$
\begin{align}\label{LNbfFormula2}
BF(M_0,M_1) = e^{\cal{K}}.
\end{align}
Alternatively, you can directly interpret the strength of evidence in favor of $M_0$ in log space by comparing the values of $\cal{K}$ to the appropriate scale (Table ref{bftable}, second column).
In this case, we evaluate $\cal{K}$ in favor of model $M_0$ against model $M_1$ so that:
\begin{center}
\begin{tabular}{l}
if $\mathcal{K} > 1$, model $M_0$ is preferred\\
if $\mathcal{K} < -1$, model $M_1$ is preferred.
\end{tabular}
\end{center}
Thus, values of $\mathcal{K}$ around 0 indicate that there is no preference for either model. 

Using the values you entered in Table \ref{tab:ml_cytb} and equation \ref{LNbfFormula}, calculate the ln-Bayes factors (using $\mathcal{K}$) for each model comparison. 
Enter your answers in Table \ref{bfTable} using the stepping-stone and the path-sampling estimates of the marginal log-likelihoods. 

\begin{Form}
\begin{table}[h!]
\centering
\caption{\small Bayes factor calculation$^*$.}
\resizebox{\textwidth}{!}{%  
\begin{tabular}{l c c c c c c c c c}
\hline
\textbf{Model comparison} & $M_1$ & $M_2$ & $M_3$ & $M_4$ & $M_5$ & $M_6$ & $M_7$ & $M_8$ & $M_9$ \\ 
\hline
$M_1$ & - & \TextField[name=bf12,backgroundcolor={.85 .85 .85},color={0 0 1},height=4ex]{}  & \TextField[name=bf13,backgroundcolor={.85 .85 .85},color={0 0 1},height=4ex]{}  & \TextField[name=bf14,backgroundcolor={.85 .85 .85},color={0 0 1},height=4ex]{}  & \TextField[name=bf15,backgroundcolor={.85 .85 .85},color={0 0 1},height=4ex]{}  & \TextField[name=bf16,backgroundcolor={.85 .85 .85},color={0 0 1},height=4ex]{}  & \TextField[name=bf17,backgroundcolor={.85 .85 .85},color={0 0 1},height=4ex]{}  & \TextField[name=bf18,backgroundcolor={.85 .85 .85},color={0 0 1},height=4ex]{}  & \TextField[name=bf19,backgroundcolor={.85 .85 .85},color={0 0 1},height=4ex]{} \\
\hline
$M_2$ & \TextField[name=bf21,backgroundcolor={.85 .85 .85},color={1 0 0},height=4ex]{}  & - & \TextField[name=bf23,backgroundcolor={.85 .85 .85},color={0 0 1},height=4ex]{}  & \TextField[name=bf24,backgroundcolor={.85 .85 .85},color={0 0 1},height=4ex]{}  & \TextField[name=bf25,backgroundcolor={.85 .85 .85},color={0 0 1},height=4ex]{}  & \TextField[name=bf26,backgroundcolor={.85 .85 .85},color={0 0 1},height=4ex]{}  & \TextField[name=bf27,backgroundcolor={.85 .85 .85},color={0 0 1},height=4ex]{}  & \TextField[name=bf28,backgroundcolor={.85 .85 .85},color={0 0 1},height=4ex]{}  & \TextField[name=bf29,backgroundcolor={.85 .85 .85},color={0 0 1},height=4ex]{} \\
\hline
$M_3$ & \TextField[name=bf31,backgroundcolor={.85 .85 .85},color={1 0 0},height=4ex]{}  & \TextField[name=bf32,backgroundcolor={.85 .85 .85},color={0 0 1},height=4ex]{}  & -  & \TextField[name=bf34,backgroundcolor={.85 .85 .85},color={0 0 1},height=4ex]{}  & \TextField[name=bf35,backgroundcolor={.85 .85 .85},color={0 0 1},height=4ex]{}  & \TextField[name=bf36,backgroundcolor={.85 .85 .85},color={0 0 1},height=4ex]{}  & \TextField[name=bf37,backgroundcolor={.85 .85 .85},color={0 0 1},height=4ex]{}  & \TextField[name=bf38,backgroundcolor={.85 .85 .85},color={0 0 1},height=4ex]{}  & \TextField[name=bf39,backgroundcolor={.85 .85 .85},color={0 0 1},height=4ex]{} \\
\hline
$M_4$ & \TextField[name=bf41,backgroundcolor={.85 .85 .85},color={1 0 0},height=4ex]{}  & \TextField[name=bf42,backgroundcolor={.85 .85 .85},color={0 0 1},height=4ex]{}  & \TextField[name=bf43,backgroundcolor={.85 .85 .85},color={0 0 1},height=4ex]{}  &-  & \TextField[name=bf45,backgroundcolor={.85 .85 .85},color={0 0 1},height=4ex]{}  & \TextField[name=bf46,backgroundcolor={.85 .85 .85},color={0 0 1},height=4ex]{}  & \TextField[name=bf47,backgroundcolor={.85 .85 .85},color={0 0 1},height=4ex]{}  & \TextField[name=bf48,backgroundcolor={.85 .85 .85},color={0 0 1},height=4ex]{}  & \TextField[name=bf49,backgroundcolor={.85 .85 .85},color={0 0 1},height=4ex]{} \\
\hline
$M_5$ & \TextField[name=bf51,backgroundcolor={.85 .85 .85},color={1 0 0},height=4ex]{}  & \TextField[name=bf52,backgroundcolor={.85 .85 .85},color={0 0 1},height=4ex]{}  & \TextField[name=bf53,backgroundcolor={.85 .85 .85},color={0 0 1},height=4ex]{}  & \TextField[name=bf54,backgroundcolor={.85 .85 .85},color={0 0 1},height=4ex]{}  & -  & \TextField[name=bf56,backgroundcolor={.85 .85 .85},color={0 0 1},height=4ex]{}  & \TextField[name=bf57,backgroundcolor={.85 .85 .85},color={0 0 1},height=4ex]{}  & \TextField[name=bf58,backgroundcolor={.85 .85 .85},color={0 0 1},height=4ex]{}  & \TextField[name=bf59,backgroundcolor={.85 .85 .85},color={0 0 1},height=4ex]{} \\
\hline
$M_6$ & \TextField[name=bf61,backgroundcolor={.85 .85 .85},color={1 0 0},height=4ex]{}  & \TextField[name=bf62,backgroundcolor={.85 .85 .85},color={0 0 1},height=4ex]{}  & \TextField[name=bf63,backgroundcolor={.85 .85 .85},color={0 0 1},height=4ex]{}  & \TextField[name=bf64,backgroundcolor={.85 .85 .85},color={0 0 1},height=4ex]{}  & \TextField[name=bf65,backgroundcolor={.85 .85 .85},color={0 0 1},height=4ex]{}  & - & \TextField[name=bf67,backgroundcolor={.85 .85 .85},color={0 0 1},height=4ex]{}  & \TextField[name=bf68,backgroundcolor={.85 .85 .85},color={0 0 1},height=4ex]{}  & \TextField[name=bf69,backgroundcolor={.85 .85 .85},color={0 0 1},height=4ex]{} \\
\hline
$M_7$ & \TextField[name=bf71,backgroundcolor={.85 .85 .85},color={1 0 0},height=4ex]{}  & \TextField[name=bf72,backgroundcolor={.85 .85 .85},color={0 0 1},height=4ex]{}  & \TextField[name=bf73,backgroundcolor={.85 .85 .85},color={0 0 1},height=4ex]{}  & \TextField[name=bf74,backgroundcolor={.85 .85 .85},color={0 0 1},height=4ex]{}  & \TextField[name=bf75,backgroundcolor={.85 .85 .85},color={0 0 1},height=4ex]{}  & \TextField[name=bf76,backgroundcolor={.85 .85 .85},color={0 0 1},height=4ex]{}  &-  & \TextField[name=bf78,backgroundcolor={.85 .85 .85},color={0 0 1},height=4ex]{}  & \TextField[name=bf79,backgroundcolor={.85 .85 .85},color={0 0 1},height=4ex]{} \\
\hline
$M_8$ & \TextField[name=bf81,backgroundcolor={.85 .85 .85},color={1 0 0},height=4ex]{}  & \TextField[name=bf82,backgroundcolor={.85 .85 .85},color={0 0 1},height=4ex]{}  & \TextField[name=bf83,backgroundcolor={.85 .85 .85},color={0 0 1},height=4ex]{}  & \TextField[name=bf84,backgroundcolor={.85 .85 .85},color={0 0 1},height=4ex]{}  & \TextField[name=bf85,backgroundcolor={.85 .85 .85},color={0 0 1},height=4ex]{}  & \TextField[name=bf86,backgroundcolor={.85 .85 .85},color={0 0 1},height=4ex]{}  & \TextField[name=bf87,backgroundcolor={.85 .85 .85},color={0 0 1},height=4ex]{}  & - & \TextField[name=bf89,backgroundcolor={.85 .85 .85},color={0 0 1},height=4ex]{} \\
\hline
$M_9$ & \TextField[name=bf91,backgroundcolor={.85 .85 .85},color={1 0 0},height=4ex]{}  & \TextField[name=bf92,backgroundcolor={.85 .85 .85},color={0 0 1},height=4ex]{}  & \TextField[name=bf93,backgroundcolor={.85 .85 .85},color={0 0 1},height=4ex]{}  & \TextField[name=bf94,backgroundcolor={.85 .85 .85},color={0 0 1},height=4ex]{}  & \TextField[name=bf95,backgroundcolor={.85 .85 .85},color={0 0 1},height=4ex]{}  & \TextField[name=bf96,backgroundcolor={.85 .85 .85},color={0 0 1},height=4ex]{}  & \TextField[name=bf97,backgroundcolor={.85 .85 .85},color={0 0 1},height=4ex]{}  & \TextField[name=bf98,backgroundcolor={.85 .85 .85},color={0 0 1},height=4ex]{}  & - \\
\hline
{\footnotesize{$^*$you can edit this table}}\\
\end{tabular}}
\label{bfTable}
\end{table}
\end{Form}



\newpage
\section{For your consideration...}
In this tutorial you have learned how to use \RevBayes~to assess the \emph{relative} fit of a pool of candidate substitution models to a given sequence alignment.
Typically, once we have identified the ``best'' substitution model for our alignment, we would then proceed to use this model for inference.
Technically, this is a decision to condition our inferences on the selected model, which explicitly assumes that it provides a reasonable description of the process that gave rise to our data.
However, there are several additional issues to consider before proceeding along these lines, which we briefly mention below.
%considerations cases where conditioning on the preferred model may be problematic.
%On possible scenario is related to uncertainty in the choice of substitution model, and the other is related to the \emph{absolute} fit of the chosen model to our data. 

\subsection{Accommodating Process Heterogeneity}
In the analyses that we performed in this tutorial, we assumed that all sites in an alignment evolved under an identical substitution process.
It is well established that this assumption is likely to be violated in real datasets.
Various aspects of the substitution process---the stationary frequencies, exchangeability rates, degree of ASRV, etc.---may vary across sites of our sequence alignment.
For example, the nature of the substitution process may vary between codon positions of protein-coding genes, between stem and loop regions of ribosomal genes, or between different gene and/or genomic regions. 
It is equally well established that failure to accommodate this \textit{process heterogeneity}---variation in the nature of the substitution process across the alignment---will cause biased estimates of the tree topology, branch lengths and other phylogenetic model parameters.
We can accommodate process heterogeneity by adopting a \emph{mixed-model} approach, where two or more subsets of sites are allowed to evolve under distinct substitution processes.
We will demonstrate how to specify---and select among---alternative mixed models using \RevBayes~in a separate tutorial, RB\_PartitionedData\_Tutorial.

\subsection{Assessing Model Adequacy}
In this tutorial, we used Bayes factors to assess the fit of various substitution models to our sequence data, effectively establishing the \emph{relative} rank of the candidate models.
Even if we have successfully identified the very best model from the pool of candidates, however, the preferred model may nevertheless be woefully inadequate in an \emph{absolute} sense. 
For this reason, it is important to consider \emph{model adequacy}: whether a given model provides a reasonable description of the process that gave rise to our sequence data. 
We can assess the absolute fit of a model to a given dataset using \emph{posterior predictive simulation}.
This approach is based on the following premise: if the candidate model provides a reasonable description of the process that gave rise to our dataset, then we should be able to generate data under this model that resemble our observed data.
%should be able to , where we assess the ability of the candidate model to predict datasets that are similar to our observed data. 
We will demonstrate how to assess model adequacy using \RevBayes~in a separate tutorial, RB\_ModelAdequacy\_Tutorial.

\subsection{Accommodating Model Uncertainty}
Even when we have carefully assessed the relative and absolute fit of candidate models to our dataset, it may nevertheless be unwise to condition our inference on the best model.
%correctly ranked the relative fit of a pool of candidate models to our dataset, and assessed adequacy of the best model, 
Imagine, for example, that there are several (possibly many) alternative models that provide a similarly good fit to our given dataset.
%This is apt to become increasingly plausible as the size of the candidate model pool continues to grow.
%[Consider, for example, that there are 203 members of GTR substitution model family.]   
In such scenarios, conditioning inference on \textit{any} single model (even the `best') ignores uncertainty in the chosen model, which will cause estimates to be biased.
This is the issue of \emph{model uncertainty}.
%Model uncertainty can be addressed by means of \textit{model averaging}---parameters are estimated under each of the candidate models, and then summarized as the weighted average over the candidate models, where the weighting is based on the probability of each candidate model.
The Bayesian framework provides a natural approach for accommodating model uncertainty by means of \textit{model averaging}; we simply adopt the perspective that models (like standard parameters) are random variables, and integrate the inference over the distribution of candidate models.
We will demonstrate how to accommodate model uncertainty using \RevBayes~in a separate tutorial, RB\_ModelAveraging\_Tutorial.


%If the fit of the data to the best model is substantially better than the fit to the other models, conditioning on the best model (i.e., making it a fixed assumption of the analysis) may be justifiable. 
%Conversely, if the fit of the data to the best model is only marginally better than the fit to the other models, model averaging may be necessary. 
%The Bayesian perspective on this problem is ``If there is uncertainty in model specification, let's integrate it out!'' 
%Specifically, for a given model, we treat each model parameter as a random variable and average inferences over the prior probability density for that parameter to accommodate for uncertainty in the parameters. 
%The idea here is to treat the MODEL as a random variable, and average estimates over the expanse of model space, while simultaneously averaging over prior probability densities for each parameter of each model.
%Because different models may differ in the number of free parameters, this entails changes in the dimensionality of parameter space, which is accommodated by using a trans-dimensional MCMC algorithm that jumps between dimensions in model space. 
%The proportion of time that the RJ MCMC visits a given model is an approximation of the posterior probability of that model.
%
%
%
%Bayesian model averaging has been implemented for various phylogenetic problems using reversible-jump MCMC, where the chain integrates over the joint prior probability density of a given model in the usual manner, but also `jumps' between models, visiting each model in proportion to its marginal probability.
%For example, rjMCMC has been used to average over the pool of $203$ substitution models corresponding to all possible combinations of the six exchangeability parameters \citep{huelsenbeck04}.
%Analyses of empirical datasets demonstrate that the credible set typically contains many substitution models.
%Moreover, accommodating this source of uncertainty has been shown to impact parameters of interest, providing less biased estimates of the topological variance and corresponding clade probabilities \citep {huelsenbeck04}.
%Similar results have been shown for methods that average over relaxed-clock models \citep{li10}:
%typically, many relaxed-clock model have considerable marginal posterior probability ({\it i.e.}, the credible set of models contains several models), and averaging over models results in better estimates of uncertainty in divergence-time estimates \citep{li10}
%We will demonstrate how to accommodate model uncertainty using \RevBayes~in a separate tutorial, RB\_ModelAveraging\_Tutorial.



%
%\newpage
%\section{Partitioned Analysis}
%Now that you have identified the best substitution model for each of the two genes, we will run a joint analysis under both genes: a partitioned analysis.




\newpage
\vspace{5cm}
Questions about this tutorial can be directed to: \\\vspace{-10mm}
\begin{itemize}
\item Tracy Heath (email: \href{mailto:tracyh@berkeley.edu}{tracyh@berkeley.edu}) \\\vspace{-8mm}
\item Sebastian H\"{o}hna (email: \href{mailto:sebastian.hoehna@gmail.com}{sebastian.hoehna@gmail.com}) \\\vspace{-8mm}
\item Michael Landis (email: \href{mailto:mlandis@berkeley.edu}{mlandis@berkeley.edu}) \\\vspace{-8mm} 
\item Brian R. Moore (email: \href{mailto:brianmoore@ucdavis.edu}{brianmoore@ucdavis.edu}) \\\vspace{-8mm}
\end{itemize}


\bibliographystyle{sysbio}
\bibliography{\ResourcePath refs}


